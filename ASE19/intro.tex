\section{Introduction}
\label{sec:intro}

Over the past decade, code search has emerged as an interesting, but
challenging, topic to both industry and research communities. Various code search
techniques have been proposed in the
literature~\cite{bajracharya2009sourcerer,reiss2009semantics,lazzarini2009applying,mcmillan2012exemplar}, and
some search code engines have been implemented and are, or were, publicly
available~\cite{googlesearch, github,codase,krugle,ohloh,searchcode}. All
of these code search engines take some specification as input (a
query, a code fragment, or a test) and retrieve pieces of code that try
to match that specification.

% motivation example

This work addresses the code search problem from a different angle. Consider
the following scenario. A programmer is implementing a Java method for
file decompression; this method gets as input the path to a zip file
unpacks all files within the zip into a target directory, as shown in 
Listing~\ref{lst:mot-query}. This piece of code is sufficient for 
a simple program task of unpacking a zip file. However, in pratice, the programmer 
may undertake a more complex programming task where unzipping a file is a small,
integral part. Therefore, the programmer may also want to know what else may be
related to this functionality. %, in case they will also need that in the future. 
The programmer may also want to have a comprehensive understanding of the collection of related
functions for zip file manipulation, in general. If an additional
functionality often co-occurs with unzipping, the programmer may want to add it
to her own project as needed. Listing \ref{lst:mot-related} shosws an example of this kind of additional 
functionalities---a method that zips a list of files from a folder into the target zip file. Unzipping
and zipping are two kinds of file manipulation in the opposite direction. 
Though these two functions work independently, they are often implemented together in a codebase to facilitate
possible needs from any direction. We consider the zip method and the unzip method {\em related} to each other, or {\em complementary code fragement}, to be more specifically.

\begin{figure*}[!t]
\begin{minipage}[t]{0.5\linewidth}
\begin{lstlisting}[style=MyJavaSmallStyle, caption={Query code\todo{Simply listing 1 to a similar length as listing 2}}, label={lst:mot-query}]
public static boolean unpackZip(String path, String zipname, String targetDirectory) {
	InputStream is;
	ZipInputStream zis;
	try {
		String filename;
		is = new FileInputStream(path + zipname);
		zis = new ZipInputStream(new BufferedInputStream(is));
		ZipEntry ze;
		byte[] buffer = new byte[1024];
		int count;

		while ((ze = zis.getNextEntry()) != null) {
			filename = ze.getName();

			if (ze.isDirectory()) {
				File fmd = new File(targetDirectory + filename);
				fmd.mkdirs();
				continue;
			}

			FileOutputStream fout = new FileOutputStream(targetDirectory + filename);

			while ((count = zis.read(buffer)) != -1) {
				fout.write(buffer, 0, count);
			}

			fout.close();
			zis.closeEntry();
		}

		zis.close();
	} catch (IOException e) {
		e.printStackTrace();
		return false;
	}

	return true;
}
\end{lstlisting}
\end{minipage}
%
\begin{minipage}[t]{0.5\linewidth}
\begin{lstlisting}[style=MyJavaSmallStyle, caption={Recommended related code}, label={lst:mot-related}]
public static void zip(String baseFolder, List<File> files, String zipFile) {
	try  {
		BufferedInputStream origin = null;
		FileOutputStream dest = new FileOutputStream(zipFile);

		ZipOutputStream out = new ZipOutputStream(new BufferedOutputStream(dest));
		byte data[] = new byte[BUFFER];

		for (File file : files) {
			FileInputStream fi = new FileInputStream(file);
			origin = new BufferedInputStream(fi, BUFFER);
			String relativeFileName = file.getAbsolutePath().replace(baseFolder + File.separator , """");
			ZipEntry entry = new ZipEntry(relativeFileName);
			out.putNextEntry(entry);
			int count;
			while ((count = origin.read(data, 0, BUFFER)) != -1) {
				out.write(data, 0, count);
			}
			origin.close();
		}

		out.close();
	} catch(Exception e) {
		e.printStackTrace();
	}

}
\end{lstlisting}
\end{minipage}
\end{figure*}
% code search and code completion

Code-to-code search engines could potentially be used to retrieve releveant code examples
to a given code query~\cite{kim2018Facoy, krugle, searchcode}. 
However, these techniques aim at finding syntactically or semantically 
similar code fragments only, without considering about auxiliary or complementary functionality. 
For instance, given an unzip function, they cannot find a complementary zip function, since neither
the implementation nor the functionality of these two operations are similar. However, as discussed earlier,
such complementary methods often co-occur in the same source file or the same codebase, which serves as 
an interesting property to exploit for recommending relevant code examples.

Pattern-based code completion tools~\cite{nguyen2009groum,
  nguyen2012grapacc} also recommend completing code for a given code
query. They do so by mining common API usage patterns from a large
code corpus. For a given partial snippet as query, if it matches a
prefix of a mined pattern, the tool recommends the rest of the pattern
for completion. Again, such tools only work for the mined patterns;
that is, they do not recommend code outside the mined patterns.

For both code search engines and pattern-based code completion tools,
the retrieved code snippets may have extra lines of code with more
functionality, but they are not designed to search for commonly used
additional code.

The goal of our work is to support the search needs shown in Listings
\ref{lst:mot-query} and \ref{lst:mot-related}. In this paper, we
describe {\tool}, a recommendation engine for related code. Given a
code snippet as input query and a large corpus of code containing
millions of code fragments, {\tool} returns a set of recommended code
fragments such that:
\begin{itemize}
	\item the recommended code fragments co-occur with similar counterparts of the input query.
	\item the recommended code fragments are ranked by their
          relevance as complements to the input query.
\end{itemize}


For the time being, we focus on method-level code fragments written in
Java. Both the query snippet and the recommended related code
fragments are Java methods. {\tool} works by first tokenizing the
query and all methods in the code corpus. It then uses token
similarity to detect similar counterparts to the query in the code
corpus. We delegate this process to a clone detection tool,
SourcererCC~\cite{sajnani2016sourcerercc}. Finally, {\tool} recognizes
other methods which co-occur with these similar counterparts as
candidate related methods.

% advantages of CodeAid
{\tool} has the following properties:
\begin{itemize}
	\item It retrieves functionality groups, which the user may want to implement together with the input query.
	\item It has a clustering algorithm on top of co-occurrence to provide ranking.
	\item It is not restricted to any programming language. The Java parser is only used to chunk the file into methods, it can be replaced by the parser from any languages as needed.
	\item It has flexible granularity level. We can chunk the files into blocks of any size. All similarity comparison processes are token-based, which means as long as we have the token list representing the block, it does not matter what size the block is.
	\item It is fast enough to be used in real time. The most
          time-consuming part is similar code detection. However,
          generating the indexes is a one-time task and can be done
          before any query is processed.	
\end{itemize}

In order to evaluate {\tool}, we use two datasets: (1) a query dataset
consisting of 21,207 Java code snippets collected from Stack Overflow
(SO), and (2) a large Java code base consisting of 50,826 projects which 
have at least five starts collected from GitHub, with over 5.8M distinct 
Java files. We then run each query in the query dataset through {\tool} 
and collect the recommended fragments coming from the code base. 
We present both a quantitative and qualitative analysis of the results.
We build a chrome extension for Stack Overflow as a proof of concept and also to help with the qualitative evaluation.

The rest of the paper is organized as follows: Section
\ref{sec:approach} describes the algorithm {\tool} uses to generate
recommendations. Section \ref{sec:eval} presents the manual analysis result of recommended code by {\tool} in terms of code relevance. We also compare {\tool} with existing code search engines in this section. \ref{sec:chrome} illustrates the instantiation of {\tool} as a chrome extension as well as a use scenario in a web browser. Section \ref{sec:related} describes and Section \ref{sec:conclude} concludes the paper.
