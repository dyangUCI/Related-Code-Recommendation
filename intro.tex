\section{Introduction}
\label{sec:intro}

Over the past decade, code search has emerged as an interesting, but
challenging, topic in both research and practice. Various code search
techniques have been proposed in the
literature~\cite{sourcerer,reiss:icse2009,codegenie,exemplar:tse:2011}, and
some search code engines have been implemented and are, or were, publicly
available~\cite{google_code_search,github,codase,krugle,ohloh,searchcode}. All
of these code search engines take some specification as input (a
query, a code fragment, or a test) and retrieve pieces of code that try
to match that specification.

% motivation example

In this work, we look at code search from a different angle. Consider
the following scenario. A programmer is implementing a Java method for
file decompression; this method gets as input the path to a zip file,
and it unpacks all the files inside onto a target directory. This
method could look like Listing~\ref{lst:mot-query}. For a single piece
of functionality, such as unpacking a zip file, the code does what it
is supposed to do, but the programmer wants to know what else may be
related to this functionality, in case they will also need that in the
future. They want to have a broader idea of the family of related
functions for zip file manipulation, in general. If other extra
functionality co-occurs often with unzipping, they may want to add it
to their own project too. One example of this kind of extra
functionality is shown in Listing \ref{lst:mot-related}, a method that
zips a list of files from a folder into the target zip file. Unzipping
and zipping are two directions of file manipulation. They can work
independently, but they are often implemented together, to facilitate
possible needs from any direction. We call the zip method a
\textit{related code fragment} to the unzip method.

\begin{figure*}[!t]
\begin{minipage}[t]{0.5\linewidth}
\begin{lstlisting}[style=MyJavaSmallStyle, caption={Query code\todo{Simply listing 1 to a similar length as listing 2}}, label={lst:mot-query}]
public static boolean unpackZip(String path, String zipname, String targetDirectory) {
	InputStream is;
	ZipInputStream zis;
	try {
		String filename;
		is = new FileInputStream(path + zipname);
		zis = new ZipInputStream(new BufferedInputStream(is));
		ZipEntry ze;
		byte[] buffer = new byte[1024];
		int count;

		while ((ze = zis.getNextEntry()) != null) {
			filename = ze.getName();

			if (ze.isDirectory()) {
				File fmd = new File(targetDirectory + filename);
				fmd.mkdirs();
				continue;
			}

			FileOutputStream fout = new FileOutputStream(targetDirectory + filename);

			while ((count = zis.read(buffer)) != -1) {
				fout.write(buffer, 0, count);
			}

			fout.close();
			zis.closeEntry();
		}

		zis.close();
	} catch (IOException e) {
		e.printStackTrace();
		return false;
	}

	return true;
}
\end{lstlisting}
\end{minipage}
%
\begin{minipage}[t]{0.5\linewidth}
\begin{lstlisting}[style=MyJavaSmallStyle, caption={Recommended related code}, label={lst:mot-related}]
public static void zip(String baseFolder, List<File> files, String zipFile) {
	try  {
		BufferedInputStream origin = null;
		FileOutputStream dest = new FileOutputStream(zipFile);

		ZipOutputStream out = new ZipOutputStream(new BufferedOutputStream(dest));
		byte data[] = new byte[BUFFER];

		for (File file : files) {
			FileInputStream fi = new FileInputStream(file);
			origin = new BufferedInputStream(fi, BUFFER);
			String relativeFileName = file.getAbsolutePath().replace(baseFolder + File.separator , """");
			ZipEntry entry = new ZipEntry(relativeFileName);
			out.putNextEntry(entry);
			int count;
			while ((count = origin.read(data, 0, BUFFER)) != -1) {
				out.write(data, 0, count);
			}
			origin.close();
		}

		out.close();
	} catch(Exception e) {
		e.printStackTrace();
	}

}
\end{lstlisting}
\end{minipage}
\end{figure*}
% code search and code completion

Code-to-code search engines could potentially be used to get code
recommendations for our unzipping example. For example, code-to-code search tools
~\cite{kim2018Facoy, krugle, searchcode} could take the code snippet
as query and retrieve similar code snippets from their code
corpora. However, such code-to-code search tools aim at finding
similar code, not extra functionality, so they retrieve several
versions of unzipping files. 

Pattern-based code completion tools~\cite{nguyen2009groum,
  nguyen2012grapacc} also recommend completing code for a given code
query. They do so by mining common API usage patterns from a large
code corpus. For a given partial snippet as query, if it matches a
prefix of a mined pattern, the tool recommends the rest of the pattern
for completion. Again, such tools only work for the mined patterns;
that is, they do not recommend code outside the mined patterns.

For both code search engines and pattern-based code completion tools,
the retrieved code snippets may have extra lines of code with more
functionality, but they are not designed to search for commonly used
additional code.

The goal of our work is to support the search needs shown in Listings
\ref{lst:mot-query} and \ref{lst:mot-related}. In this paper, we
describe {\tool}, a recommendation engine for related code. Given a
code snippet as input query and a large corpus of code containing
millions of code fragments, {\tool} returns a set of recommended code
fragments such that:
\begin{itemize}
	\item the recommended code fragments co-occur with similar counterparts of the input query.
	\item the recommended code fragments are ranked by their
          relevance as complements to the input query.
\end{itemize}


For the time being, we focus on method-level code fragments written in
Java. Both the query snippet and the recommended related code
fragments are Java methods. {\tool} works by first tokenizing the
query and all methods in the code corpus. It then uses token
similarity to detect similar counterparts to the query in the code
corpus. We delegate this process to a clone detection tool,
SourcererCC~\cite{sajnani2016sourcerercc}. Finally, {\tool} recognizes
other methods which co-occur with these similar counterparts as
candidate related methods.

% advantages of CodeAid
{\tool} has the following properties:
\begin{itemize}
	\item It retieves functionality groups, which the user may want to implement together with the input query.
	\item It has a clustering algorithm on top of co-occurrence to provide ranking.
	\item It is not restricted to any programming language. The Java parser is only used to chunk the file into methods, it can be replaced by the parser from any languages as needed.
	\item It has flexible granularity level. We can chunk the files into blocks of any size. All similarity comparison processes are token-based, which means as long as we have the token list representing the block, it does not matter what size the block is.
	\item It is fast enough to be used in real time. The most
          time-consuming part is similar code detection. However,
          generating the indexes is a one-time task and can be done
          before any query is processed.	
\end{itemize}

In order to evaluate {\tool}, we use two datasets: (1) a query dataset
consisting of 11,110 Java code snippets collected from Stack Overflow
(SO), and (2) a large Java code base consisting of 50,826 projects which 
have at least five starts collected from GitHub, with over 5.8M distinct 
Java files. We then run each query in the query dataset through {\tool} 
and collect therecommended fragments coming from the code base. 
We present both a quantitative and qualitative analysis of the results.

The rest of the paper is organized as follows: Section
\ref{sec:approach}, we describe the algorithm {\tool} uses to generate
recommendations. In Section \ref{sec:eval}, we manually analyze how
relevant {\tool} recommendations are, and what kinds of relevance they
provide. We also compare {\tool} results with those from code search
engines in this section. Section \ref{sec:related} presents the
related work. Finally, Section \ref{sec:conclude} concludes the paper.
