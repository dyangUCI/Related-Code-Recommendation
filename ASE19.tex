\documentclass[conference]{IEEEtran}
\IEEEoverridecommandlockouts
% The preceding line is only needed to identify funding in the first footnote. If that is unneeded, please comment it out.
\usepackage{cite}
\usepackage{amsmath,amssymb,amsfonts}
\usepackage{algorithmic}
\usepackage[ruled,norelsize]{algorithm2e}
\usepackage{graphicx}
\usepackage{textcomp}
\usepackage{color}
\usepackage{xcolor}
\usepackage{listings}
\usepackage{booktabs}
\usepackage{soul}
\usepackage{url}

\newcommand{\ttt}[1]{\tt\small{#1}}
\newcommand{\tool}{{\ttt CodeAid}}
\lstset{
	% numbers=left,
	% numberstyle=\tiny,
	% backgroundcolor=\color{light-gray},
	basicstyle=\scriptsize\ttfamily,
	breaklines=true,
	breakatwhitespace=true,
	captionpos=b,
	columns=flexible,
	escapeinside={(*@}{@*)},
	frame=tb,
	framerule=0.6pt,
	% xleftmargin=\parindent,
	% xrightmargin=\parindent,
	language=Java,
	% numbersep=5pt,
	showstringspaces=false
}
\newcommand{\explanation}[1]{
	\parbox{0.45\textwidth}{
		\rule{0.4\textwidth}{0.1pt}
		\vspace*{0.5em} \\
		{#1}
	}
}

\def\BibTeX{{\rm B\kern-.05em{\sc i\kern-.025em b}\kern-.08em
    T\kern-.1667em\lower.7ex\hbox{E}\kern-.125emX}}
\begin{document}

\newcommand\todo[1]{\textcolor{red}{TODO: #1}}

\makeatletter
\newcommand{\removelatexerror}{\let\@latex@error\@gobble}
\makeatother

\title{CodeAid: Recommending Related Code
%	\\
%{\footnotesize \textsuperscript{*}Note: Sub-titles are not captured in Xplore and
%should not be used}
%\thanks{Identify applicable funding agency here. If none, delete this.}
}

%\author{\IEEEauthorblockN{1\textsuperscript{st} Given Name Surname}
%\IEEEauthorblockA{\textit{dept. name of organization (of Aff.)} \\
%\textit{name of organization (of Aff.)}\\
%City, Country \\
%email address}
%\and
%\IEEEauthorblockN{2\textsuperscript{nd} Given Name Surname}
%\IEEEauthorblockA{\textit{dept. name of organization (of Aff.)} \\
%\textit{name of organization (of Aff.)}\\
%City, Country \\
%email address}
%\and
%\IEEEauthorblockN{3\textsuperscript{rd} Given Name Surname}
%\IEEEauthorblockA{\textit{dept. name of organization (of Aff.)} \\
%\textit{name of organization (of Aff.)}\\
%City, Country \\
%email address}
%\and
%\IEEEauthorblockN{4\textsuperscript{th} Given Name Surname}
%\IEEEauthorblockA{\textit{dept. name of organization (of Aff.)} \\
%\textit{name of organization (of Aff.)}\\
%City, Country \\
%email address}
%\and
%\IEEEauthorblockN{5\textsuperscript{th} Given Name Surname}
%\IEEEauthorblockA{\textit{dept. name of organization (of Aff.)} \\
%\textit{name of organization (of Aff.)}\\
%City, Country \\
%email address}
%\and
%\IEEEauthorblockN{6\textsuperscript{th} Given Name Surname}
%\IEEEauthorblockA{\textit{dept. name of organization (of Aff.)} \\
%\textit{name of organization (of Aff.)}\\
%City, Country \\
%email address}
%}

\maketitle

\begin{abstract}
Finding similar code for a given code query can help programmers
detect situations that they had overlooked, or that they did not know
how to solve. Most code-to-code search tools aim at finding
syntactically or semantically similar code given some code of
interest. We take a different approach to code-to-code search: we want
to recommend relevant {\em auxiliary} or {\em complementary}
code. Such relevant code is not semantically similar to the code
query, but it is also important, because it works together with the
given code to accomplish a complete or related functionality.

In this paper, we describe a code recommendation technique, CodeAid, that returns
relevant code fragments in our search code base related to a code query. We
use GitHub Java projects as our search code base. Given a source code method,
we use a code clone detection tool to detect all its similar
counterparts from different GitHub files, then recommend other related
methods, based on co-occurrence statistics.

In order to evaluate the performance of CodeAid, we use Stack Overflow
(SO) Java code snippets as queries, and measure the precision of the
retrieved related methods. We gathered a query code base of 11,110 SO
Java code snippets. For each of these, CodeAid returns their related
code fragments. Then we manually evaluate the relevance of the
recommended GitHub code. Using this methodology, CodeAid has a
precision of 75.6\%. We also categorize the relationship between the
recommended code and the SO query snippet, in order to shed light on
how CodeAid can be useful in practice.

\end{abstract}

\begin{IEEEkeywords}
code recommendation, related code, code search, mining software repositories
\end{IEEEkeywords}

\section{Introduction}
\label{sec:intro}
Over the past decade, code search has emerged as an interesting, but
challenging, topic to both industry and research communities. Various code search
techniques have been proposed in the
literature~\cite{bajracharya2009sourcerer,reiss2009semantics,lazzarini2009applying,mcmillan2012exemplar}, and
some code search engines have been implemented and are, or were, publicly
available~\cite{googlesearch, github,codase,krugle,ohloh,searchcode}. Code search engines 
take some specification as input (e.g., a keyword description, a code fragment, or a test) and 
recommend pieces of code that match the given specification based on some forms of ``similarity'' measurement. Prior work has shown that by identifying and contrasting similar code, programmers could quickly understand the gist of implementing a function and explore potential variations to write more complete and robust code~\cite{zhang2019analyzing, luan2019aroma}. For instance, by inspecting variations in similar code found in GitHub, programmers are able to identify critical code parts such as safety checks and exception handling logics that are missed in a given code example~\cite{zhang2019analyzing}.

% motivation example

This work explores the opportunities of searching relevant code beyond similarity. Consider
the following scenario. A programmer is implementing a Java method for
file decompression. A code search engine may recommend a code example as as shown in 
Listing~\ref{lst:mot-query} in Figure~\ref{fig:example}, which takes the path to a zip file as input and
unpacks all files within the zip into a target directory.  This piece of code is sufficient for 
a simple program task of unpacking a zip file. However, in practice, the programmer 
may undertake a more complex programming task where unzipping a file is a small,
integral part. Therefore, the programmer may also want to know what else may be
related to this functionality. 
%The programmer may also want to have a comprehensive understanding of the collection of related
%functions for zip file manipulation, in general.
 If an additional functionality often co-occurs with unzipping, the programmer may want to add it
to her own project as needed. 
For instance, Listing \ref{lst:mot-related} shows an example of this kind of additional 
functionalities---a method that zips a list of files from a folder into the target zip file. Unzipping
and zipping are two kinds of file manipulation in the opposite direction. 
Though these two functions work independently, they are often implemented together in a code base to complement each other. We consider the zip method and the unzip method {\em related} to each other, or {\em complementary code fragment}, to be more specifically.

\begin{figure*}[!t]
\begin{minipage}[t]{0.5\linewidth}
\begin{lstlisting}[style=MyJavaSmallStyle, caption={Query code: unpacking a zip file}, label={lst:mot-query}]
public static boolean unpackZip(String path, String zipname, String targetDirectory) {
	try {
		InputStream is = new FileInputStream(path + zipname);
		ZipInputStream zis = new ZipInputStream(new BufferedInputStream(is));
		byte[] buffer = new byte[1024];
		int count;
		while ((ZipEntry ze = zis.getNextEntry()) != null) {
			String filename = ze.getName();
			if (ze.isDirectory()) {
				File fmd = new File(targetDirectory + filename);
				fmd.mkdirs();
				continue;
			}
			FileOutputStream fout = new FileOutputStream(targetDirectory + filename);
			while ((count = zis.read(buffer)) != -1) {
				fout.write(buffer, 0, count);
			}
			fout.close();
			zis.closeEntry();
		}
		zis.close();
	} catch (IOException e) {
		e.printStackTrace();
		return false;
	}
	return true;
}
\end{lstlisting}
\end{minipage}
%
\begin{minipage}[t]{0.5\linewidth}
\begin{lstlisting}[style=MyJavaSmallStyle, caption={Related code: zipping a file}, label={lst:mot-related}]
public static void zip(String baseFolder, List<File> files, String zipFile) {
	try  {
		BufferedInputStream origin = null;
		FileOutputStream dest = new FileOutputStream(zipFile);

		ZipOutputStream out = new ZipOutputStream(new BufferedOutputStream(dest));
		byte data[] = new byte[BUFFER];

		for (File file : files) {
			FileInputStream fi = new FileInputStream(file);
			origin = new BufferedInputStream(fi, BUFFER);
			String relativeFileName = file.getAbsolutePath().replace(baseFolder + File.separator , """");
			ZipEntry entry = new ZipEntry(relativeFileName);
			out.putNextEntry(entry);
			int count;
			while ((count = origin.read(data, 0, BUFFER)) != -1) {
				out.write(data, 0, count);
			}
			origin.close();
		}
		out.close();
	} catch(Exception e) {
		e.printStackTrace();
	}
}
\end{lstlisting}
\end{minipage}
\caption{An example of recommending relevant code that complements desired functionality}
\label{fig:example}
\end{figure*}
% code search and code completion
 
Code-to-code search engines could be leveraged to identify related code given a code 
fragment of interest~\cite{kim2018Facoy, krugle, searchcode}. 
However, these techniques find syntactically or semantically similar code fragments 
only, without considering about auxiliary or complementary functionality. 
For instance, given an unzip function, they cannot find a complementary zip function, since neither
the implementation nor the functionality of these two operations are similar. 
Currently,  there is also limited understanding about what other kinds of relevant code beyond similar code
may exist in practice and thus should be recommended modern code search engines. 

%Pattern-based code completion tools~\cite{nguyen2009groum,
%  nguyen2012grapacc} also recommend completing code for a given code
%query. They do so by mining common API usage patterns from a large
%code corpus. For a given partial snippet as query, if it matches a
%prefix of a mined pattern, the tool recommends the rest of the pattern
%for completion. Again, such tools only work for the mined patterns;
%that is, they do not recommend code outside the mined patterns.

%For both code search engines and pattern-based code completion tools,
%the retrieved code snippets may have extra lines of code with more
%functionality, but they are not designed to search for commonly used
%additional code.

Complementary methods shown in Figure~\ref{fig:example} often co-occur in the same source file or the same code base, which serves as an interesting property to exploit for recommending relevant code examples.
Manually identifying co-occurring code is tedious and time-consuming, 
since some co-occurring code may be project-specific and thus not relevant to a given code query. 
Therefore, we first build an automated approach to identify, cluster, and rank common, 
co-occurring code given a code query.  
Our approach uses a state-of-the-art clone detector called SourcererCC~\cite{sajnani2016sourcerercc} to identify similar
counterparts (i.e., clones) in a large code corpus. Then our approach contrasts surrounding code of those clones 
and identify common code that are shared around multiple clones.

Using 21K Java code snippets from Stack Overflow
as code queries, we automatically identify relevant code of these code queries in a large corpus of  50K GitHub projects with at least five stars. 
As a result, we obtained 21K groups of similar code in GitHub. We manually inspected a random sample of 50 common, co-occurring code fragments and examined their relevancy to the original query. 
74\% of those common, co-occurring code fragments represented relevant functionality, 
which should be included in code search results. 
Furthermore, we identified three major types of relevant co-occurring code---{\em complementary}, {\em supplementary}, and {\em alternative} functions. These findings show that it is beneficial to recommend common, co-occurring code of a given code query to achieve more complete functionalities, instead of just recommending similar code. 

To further demonstrate this idea, we implement a Chrome extension called {\tool} that recommends related but non-similar code when programmers browse code examples in Stack Overflow. It is well known that programmers often search and reuse online examples during modern software development~\cite{brandt2010example, umarji2008archetypal, gallardo2009internet}. {\tool} augments this programming workflow by reminding programmers what other complementary, supplementary, or alternative functions should also be included as they copy and paste code from the Web. We compared {\tool} with a state-of-the-art code search engine called Facoy~\cite{kim2018Facoy}. Among ten sample search queries, FaCoy only identified related code for one query since the related code is similar to the code query. A general-purpose search engine, Google Search, was able to identify GitHub files that contain related code recommended by {\tool}  for half of the queries. However, Google could only retrieve full files where programmers still had to manually go through those files to identify related code. By contrast, {\tool} pinpointed where related code fragments were based on how frequently they occurred in other similar locations.
%Given a SO example, {\tool} automatically detects similar code fragments in GitHub and retrieves common code that co-occurs with these fragments. {\tool} also incorporates a clustering algorithm on top of co-occurrence to provide ranking. 
%Though {\tool} currently supports Java, it is not restricted to any programming languages. The Java parser is only used to tokenize code fragments for clone detection and can be substituted with any off-the-shelf parsers of other languages.
%The goal of our work is to explore the search needs shown in Listings
%\ref{lst:mot-query} and \ref{lst:mot-related}. In this paper, we first
%describe {\tool}, an approach for retrieving common co-occurring code fragments. 
%Given a code snippet as input query and a large corpus of code containing
%millions of code fragments, {\tool} returns a set of code
%fragments such that:
%\begin{itemize}
%	\item the code fragments co-occur with similar counterparts of the input query.
%	\item the code fragments are ranked by their commonality.
%\end{itemize}


%For the time being, we focus on method-level code fragments written in
%Java. Both the query snippet and the common co-occurring code
%fragments are Java methods. {\tool} works by first tokenizing the
%query and all methods in the code corpus. It then uses token
%similarity to detect similar counterparts to the query in the code
%corpus. We delegate this process to a clone detection tool,
%SourcererCC~\cite{sajnani2016sourcerercc}. Finally, {\tool} recognizes
%other methods which co-occur with these similar counterparts as
%candidate related methods.

% advantages of CodeAid
% {\tool} has the following properties that differentiates it from traditional similarity based code search engines:
%\begin{itemize}
%	\item It retrieves functionality groups, which the user may want to implement together with the input query.
%	\item It has a clustering algorithm on top of co-occurrence to provide ranking.
%	\item It is not restricted to any programming language. The Java parser is only used to chunk the file into methods, it can be replaced by the parser from any languages as needed.
%	\item It has flexible granularity level. We can chunk the files into blocks of any size. All similarity comparison processes are token-based, which means as long as we have the token list representing the block, it does not matter what size the block is.
%	\item It is fast enough to be used in real time. The most
%          time-consuming part is similar code detection. However,
%          generating the indexes is a one-time task and can be done
%          before any query is processed.	
%\end{itemize}

%We apply our approach on two data sets: (1) a query dataset
%consisting of 21,207 Java code snippets collected from Stack Overflow
%(SO), and (2) a large Java code base consisting of 50,826 projects which 
%have at least five starts collected from GitHub, with over 5.8M distinct 
%Java files. We then run each query in the query dataset through {\tool} 
%and collect the common co-occurring fragments coming from the code base. 
%We manually inspected a random sample of 50 commonly co-occurring code fragments and examined their relevancy to the original query. 
%We found 76\% of them represent relevant functionality that should be included in code search results.
%Furthermore, we identified three major types of relevant co-occurring code---{\em complementary}, {\em supplementary}, and {\em alternative} functions.
%We build a Chrome extension for Stack Overflow to help with the qualitative evaluation and also to provide one application of recommending relative code beyond similarity.

%\todo{summarize the contributions}
In summary, this paper makes the following contribution:
\begin{itemize}
\item We present a new code search method that recommends common, co-occurring code of a given code query, rather than only recommending similar code.
\item We empirically show the prevalence of common, co-occurring code by quantifying the commonality of surrounding code of GitHub clones. We also find that the majority of such co-occurring code fragments represent meaningful functionality such as complementary, supplementary, or alternative functions, which should be recommended by modern code search engines.
\item We develop a Chrome extension called {\tool} to recommend related code during online code search and demonstrate that {\tool} is capable of recommending related code that cannot be identified by a state-of-the-art code search tool.
%\item It retrieves a large dataset of commonly co-occurring code.
%\item It provides a in-depth relevance analysis between the commonly co-occurring code with its original query. These code proved to be related and code should be considered for recommendation along with the similar ones.
%\item It implements a chrome extension as a tool for recommending related code in practice.
\end{itemize}

The rest of the paper is organized as follows: Section
\ref{sec:approach} describes the approach for generating
common, co-occurring methods, with ranking. Section \ref{sec:eval} presents the manual analysis result of common, co-occurring code in terms of its relevance to the original query. Section\ref{sec:chrome} illustrates the Chrome extension, {\tool}, as well as a use scenario of it, and Section \ref{sec:comparison} explores whether recommended code from existing search engines can provide same relevance as results from {\tool}. Section \ref{sec:related}, \ref{sec:threat}, and \ref{sec:conclude} summarize the related work, point out the threats to validity, and conclude the paper.

\section{Data Collection Approach}
\label{sec:approach}
Our approach takes a code fragment as input and searches a code corpus to identify related code fragments. Given a user-selected code fragment, we first detect its similar methods in the corpus based on syntactic similarity. Then we trace back to the containing files of these similar methods and identifies other co-occurring methods in these files. Among these co-occurring methods, we further measure each method's similarity to methods in other files, and cluster similar methods. Then we rank co-occurring methods based on the size of cluster it centers. Figure~\ref{fig:pipeline} describes the pipeline of finding commonly co-occurring code fragments in {\tool}. 


\begin{figure}
	\includegraphics[width=\linewidth]{figures/pipeline.pdf}
	\caption{The pipeline of collecting commonly co-occurring methods}
	\label{fig:pipeline}
\end{figure}

\subsection{Retrieve similar methods}
\subsubsection{Parse a code corpus}
We focus on method-level code fragments written in Java in this work. We parse all Java source files to abstract syntax trees (ASTs) and traverse the ASTs to extract all defined methods. Note that the approach is not limited to any programming language. We can switch to any other language by using its particular parser. We use the phrases code fragments and methods interchangeably in the paper.

\subsubsection{Tokenization}
Tokenization is the process of transforming source code into a bag of words. Tokenization starts from removing comments, spaces, tabs and other special characters. Then it identifies distinct tokens and count their frequencies. For each method, the result of tokenization is formatted as a list of tuples such as {\ttt (token, freq)}, where the first element is a token in the method and the second element refers to the token occurrence in the method.

We tokenize both the input code fragment and all methods in the code corpus, in preparation for the next step of finding similar pairs.

\subsubsection{Search for similar methods}
For the input code fragment, we retrieve its similar counterparts from the code corpus using a token-based clone detection tool called SoucererCC~\cite{sajnani2016sourcerercc}. By evaluating the scalability, execution time, recall and precision of SourcererCC, and comparing it to publicly available and state-of-the-art tools, SourcererCC has been shown to have both high recall and precision, and is able to scale to a large repository using a standard workstation. All of the above make SourcererCC a good candidate for building our code recommendation engine. 

%Given a similarity threshold, SoucererCC takes three steps to detect clones. First, it tokenizes a code snippet to a set of tokens. This tokenization step removes comments, whitespaces, and special characters and also counts the frequency of individual tokens. Second, SoucererCC creates a partial index of each snippet by selecting and indexing a subset of tokens based on heuristics, and builds an inverted index mapping between tokens and code snippets. Finally, SoucererCC iterates through all snippets and finds candidate clones of each snippet by querying the inverted index mapping. After retrieving the candidates, SoucererCC uses another heuristic which exploits ordering of the tokens in a snippet to verify the candidates and locate the clones. 
We use 70\% similarity threshold, because it yields the best precision and recall on multiple clone benchmarks~\cite{sajnani2016sourcerercc}. SourcererCC takes the token lists of the input code fragment and all methods in the code corpus, and returns the similar methods to the input in the code corpus. As shown in Figure \ref{fig:pipeline}, the user input has three similar counterparts in our code corpus, which are {\ttt Method A} in {\ttt File1}, {\ttt Method E} in {\ttt File2}, and {\ttt Method G} in {\ttt File3}.

\subsection{Identify co-occurring code fragments}
Given those similar code fragments identified in the previous step, we trace back to the files that contain these similar counterparts and identify co-occurring methods in the same file as a potentially related code fragment. Algorithm~\ref{alg: co-occur} gives a more formal description of the process.

\begin{figure}[h]
		
 \removelatexerror
\begin{algorithm}[H]
	\label{alg: co-occur}
	\caption{Identify co-occurring code fragments}
	\KwData{similar methods}
	\KwResult{co-occurring methods}
	initialize $resultList$\;
	\For{$m_s$ in $similarMethods$}
	{
		$ghFiles$ = traceGitHubFiles(method) \;
		\For{$f$ in $ghFiles$}
		{
			$methodsInFile$ = parse($f$)\;
			\For{$m_f$ in $methodsInFile$}
			{
				\If {$m_f$ is not $m_s$}
				{
					$resultList$.add($m_f$) \;
				}
			}
		} 
	}
\end{algorithm}
\end{figure}

{\ttt Method A, E, G} are the three similar methods detected by SourcererCC. {\ttt File 1, 2, 3} are the three GitHub files contain these similar methods respectively. 
{\ttt File1} also contains {\ttt Method B, C}, {\ttt File2} has another two methods {\ttt Method D, F}, and {\ttt Method H} is in {\ttt File3}. Therefore, {\ttt Method B, C, D, F, H} will be returned as co-occurring methods by Algorithm~\ref{alg: co-occur}.

\subsection{Clustering and Ranking}
\subsubsection{Cluster co-occurring code fragments}
We further get the token lists for those co-occurring methods identified in the previous step and remove duplicate methods. Duplication is defined as same token lists. In order to detect common co-occurring code fragments, we cluster the remaining unique oc-occurring methods
based on their token similarity. Given each method, we compute its similarity to other methods from different GitHub files. Each method will serve as the center of a cluster, we browse among other methods from different files and add similar methods to the current cluster. Algorithm~\ref{alg: clustering} describes the process.

\begin{figure}[h]
	\removelatexerror
	\begin{algorithm}[H]
		\label{alg: clustering}
		\caption{Clustering candidate related code fragments}
		\KwData{$n$ candidate related methods}
		\KwResult{clustered candidate methods}
		initialize $clusters$ = $\{X_1, X_2,..., X_n\}$\;
		\For{$m_i$ in $candidateMethods$}
		{
			$X_i$.add($m_i$) \;
			\For{$m_j$ in $candidateMethods$}
			{
				\If {$m_i$ and $m_j$ do not come from the same file}
				{
					\If {tokenSimilarity($m_i$, $m_j$) $>$ 0.7)}
					{
						$X_i$.add($m_j$);
					}
				}
			} 
		}
	\end{algorithm}
\end{figure}

For the co-occurring methods pool, {\ttt Method B, C, D, F, H}, each method will be the center of a cluster. For {\ttt Method B}, we compute token similarity with {\ttt Method D, F, H} and get two similar ones, {\ttt Method D, H}, so we add these two similar methods to the cluster, resulting in cluster size being three. Similarly, we add {\ttt Method F} to the cluster centered by {\ttt Method C} and get a cluster with size two.

\subsubsection{Screen and rank clusters by size}
After getting the candidate clusters, we keep only clusters with size being at least two. This means the center of the cluster has occurred at least twice among the GitHub files.
We rank the remaining clusters by size and return the cluster centers as our final list of commonly co-occurring code fragments with ranking. If two clusters have the same size, we will order them by the line number distance between the cluster center and the original counterparts of the query (e.g. {\ttt Method C, A}), in ascending order. We will return {\ttt Method B} first, and then {\ttt Method C}, as our common co-occurring code fragments.
\section{Dataset}
\label{sec:dataset}
In this paper, we apply our approach to Stack Overflow (SO) code snippets and GitHub projects. We use code snippets in SO as our pool of user inputs and Java projects in GitHub as our code base. We choose these two datasets not only because their popularity within the programming community, but also because they are part of a larger system of software production. The same users that rely on the hosting and management characteristics of GitHub often have difficulties and need help on the implementation of their computer programs, seek support on SO for their specific problems, or hints of solutions from ones with a degree of similarity, and return to GitHub to apply the knowledge acquired. Previous work [MSR17, ICSE19] have shown the copy-paste behaviors and possible adaptations between SO and GitHub. Our approach will help developers when browsing SO. The use scenario will be: when a user is interested in a code snippet in SO, we recommend its related code fragments from GitHub, in the purpose of showing what other code fragments he may also check. 


\subsection{GitHub}
We downloaded the Github Java projects by using the metadata provided by GHTorrent. GHTorrent is a scalable, offline mirror of data offered through the Github REST API, available to the research community as a service. It provides access to all the meta-data from GitHub, such as number of stars or commiters, main languages, time points relevant to the projects and so on.

Since GitHub has many toy projects that do not adequately reflect software engineering practices [19], we only consider GitHub projects that have at least five stars. To account for internal duplication in GitHub [20], we choose non-fork projects only and further remove duplicated GitHub files using the same file hashing method as in [20], since such file duplication may skew our analysis. As a result, we download 50,826 non-forked Java repositories with at least five stars from GitTorrent [21]. After deduplication, 5,825,727 distinct Java files remain.


\subsection{Stack Overflow}
From the SO dump taken in October 2016, we extract 312,219 answer posts that have java or android tags and also contain code snippets in the <code> markdown. We consider code snippets in answer posts only, since snippets in question posts are rarely used as examples. 

Since SO snippets are often free-standing statements with low parsable rates, we used a customized pre-processor befor tokenization. We add dummy class and method definitions, and semicolons after statements, as needed. For snippets contain multiple methods, we chunk them into individual ones. We keep only parsable SO snippets after pre-processing. Prior work finds that larger SO snippets have more meaningful clones in GitHub [26]. Hence, we choose to study SO examples with no less than 50 tokens after tokenization. We also remove duplicated examples within SO.

\subsection{Result for similar code detection}
We run SoucererCC to find all similar pairs between SO and GitHub. As a result, we get 21,207 distinct SO methods that have one or more similar code fragments in GitHub.

\subsection{Result for candidate related methods}
Within the 21,207 groups of SO snippet with GitHub files which contain similar methods to the SO snippet, we extract all co-occurred methods from these GitHub files and treat them as candidate related code fragments. Then for each candidate in each GitHub file, we retrieve its similar counterparts from other files. As a result, we get the co-occurred methods as our candidate related code fragments and for each candidate we also have the frequency of its similar counterparts. Not all groups have candidate methods and not all candidates have similar counterparts in other files, we have 10,491 SO snippets whose candidate related methods do have similar counterparts in other files, that can be taken as the candidate appears more than one files and we take this as a stronger signal for recommendation and only focus on these 10,491 groups from then on. Inside each group, we order the candidates by the number of similar counterparts and returned the ordered list as the final recommendation of related code fragments to the user.

%\section{Evaluation}
%\label{sec:eval}
%
%
%
%In this section, we describe the design of the assessment scenarios for \tool\ and report the evaluation results. Specifically, our experiments aim to address the following research questions:
%\begin{itemize}
%	\item \textbf{RQ1}: Can {\tool} retrieve related code fragments for the queries in query code base?
%	\item \textbf{RQ2}: Are the code fragments recommended by \tool\ related to the query?
%	\item \textbf{RQ3}: What kinds of related code fragments do \tool\ recommend?
%	\item \textbf{RQ4}: Can we get the recommended related code fragments from code search engines?
%\end{itemize}
%

\section{Manual analysis and categorization}
\label{sec:eval}
% Set listing style for table
\lstset{
	frame=none,
    aboveskip=0pt,
    belowskip=0pt,
    basicstyle=\tiny\ttfamily,
}
\begin{table*}\scriptsize
\caption{Complementary code examples}
\label{tab:compl-examples}

\setlength{\tabcolsep}{0.01\textwidth}
\begin{tabular}{@{}p{0.49\textwidth}p{0.49\textwidth}@{}}
\toprule
Query Code Snippet & Recommended Related Code \\
\midrule





\begin{lstlisting}
public static byte[] encrypt(final SecretKeySpec key, final byte[] iv, final byte[] message)
throws GeneralSecurityException {
	final Cipher cipher = Cipher.getInstance(AES_MODE);
	IvParameterSpec ivSpec = new IvParameterSpec(iv);
	cipher.init(Cipher.ENCRYPT_MODE, key, ivSpec);
	byte[] cipherText = cipher.doFinal(message);

	log(""cipherText"", cipherText);

	return cipherText;
}
\end{lstlisting}


&
\begin{lstlisting}
public static byte[] decrypt(final SecretKeySpec key, final byte[] iv, final byte[] decodedCipherText)
throws GeneralSecurityException {
	final Cipher cipher = Cipher.getInstance(AES_MODE);
	IvParameterSpec ivSpec = new IvParameterSpec(iv);
	cipher.init(Cipher.DECRYPT_MODE, key, ivSpec);
	byte[] decryptedBytes = cipher.doFinal(decodedCipherText);

	log(""decryptedBytes"", decryptedBytes);

	return decryptedBytes;
}
\end{lstlisting}

\vspace*{1em}
\explanation{
	\emph{Example B: Complementary method}
	\begin{itemize}
		\item The query snippet implements {\ttt encrypt} functionality for an byte array.
		\item The recommended related method decrypts a decoded byte array. 
	\end{itemize}
}

\\

\bottomrule

\begin{lstlisting}
@Override
public void onCreate(Bundle savedInstanceState) {
	super.onCreate(savedInstanceState);
	setContentView(R.layout.main);
	preferred = (TextView)findViewById(R.id.preferred);
	orientation = (TextView)findViewById(R.id.orientation);
	mgr = (SensorManager) this.getSystemService(SENSOR_SERVICE);
	accel = mgr.getDefaultSensor(Sensor.TYPE_ACCELEROMETER);
	compass = mgr.getDefaultSensor(Sensor.TYPE_MAGNETIC_FIELD);
	orient = mgr.getDefaultSensor(Sensor.TYPE_ORIENTATION);
	WindowManager window = (WindowManager)
	this.getSystemService(WINDOW_SERVICE);
	int apiLevel = Integer.parseInt(Build.VERSION.SDK);
	if(apiLevel <8) {
		mRotation = window.getDefaultDisplay().getOrientation();
	}
	else {
		mRotation = window.getDefaultDisplay().getRotation();
	}
	}
\end{lstlisting}

&
\begin{lstlisting}
@Override
protected void onPause() {
	mgr.unregisterListener(this, accel);
	mgr.unregisterListener(this, compass);
	mgr.unregisterListener(this, orient);
	super.onPause();
}
\end{lstlisting}

\vspace*{1em}
\explanation{
	\emph{Example C: Complementary method}
	\begin{itemize}
		\item The query snippet implements {\ttt onCreate} functionality for an {\ttt Android} app activity.
		\item The recommended related method implements {\ttt onPause} which does not have direct function call with the query snippet, but adds extra functionality to the activity. 
	\end{itemize}
}

\\

\bottomrule
\end{tabular}
\end{table*}

\begin{table*}\scriptsize
	\caption{Supplementary code examples}
	\label{tab:suppl-examples}
	
	\setlength{\tabcolsep}{0.01\textwidth}
	\begin{tabular}{@{}p{0.49\textwidth}p{0.49\textwidth}@{}}
		\toprule
		Query Code Snippet & Recommended Related Code \\
		\midrule
		

\begin{lstlisting}
private void queueJob(final String url, final ImageView imageView,final Drawable placeholder) {
	/* Create handler in UI thread. */
	final Handler handler = new Handler() {
	@Override
	public void handleMessage(Message msg) {
		String tag = mImageViews.get(imageView);
			if (tag != null && tag.equals(url)) {
				if (imageView.isShown())
					if (msg.obj != null) {
						imageView.setImageDrawable((Drawable) msg.obj);
					} else {
					imageView.setImageDrawable(placeholder);
					//Log.d(null, "fail " + url);
					}
				}
			}
	};

	mThreadPool.submit(new Runnable() {
		@Override
		public void run() {
			final Drawable bmp = downloadDrawable(url);
			// if the view is not visible anymore, the image will be ready for next time in cache
			if (imageView.isShown())
			{
				Message message = Message.obtain();
				message.obj = bmp;
				//Log.d(null, "Item downloaded: " + url);

				handler.sendMessage(message);
			}
		}
	});
}
\end{lstlisting}


&


\begin{lstlisting}
public void loadDrawable(final String url, final ImageView imageView) {
	imageViews.put(imageView, url);
	Drawable drawable = getDrawableFromCache(url);
	// check in UI thread, so no concurrency issues
	if (drawable != null) {
		Log.d(null, "Item loaded from cache: " + url);
		imageView.setImageDrawable(drawable);
	} else {
		imageView.setImageDrawable(placeholder);
		queueJob(url, imageView);
	}
}
\end{lstlisting}

\vspace*{1em}
\explanation{
	\emph{Example D: Supplementary method}
	\begin{itemize}
		\item The recommended related method calls the query snippet within its method body. It is a higher-level funtionality to the query.
	\end{itemize}
}

\\


\bottomrule


\begin{lstlisting}
@Override
protected void onLayout(boolean changed, int l, int t, int r, int b) {
	final int count = getChildCount();
	for (int i = 0; i < count; i++) {
		View child = getChildAt(i);
		LayoutParams lp = (LayoutParams) child.getLayoutParams();
		child.layout(lp.x+5, lp.y+5, lp.x + child.getMeasuredWidth(), lp.y + child.getMeasuredHeight());
	}
}
\end{lstlisting}

&
\begin{lstlisting}
@Override
protected LayoutParams generateLayoutParams(ViewGroup.LayoutParams p) {
	return new LayoutParams(p);
}
\end{lstlisting}

\vspace*{1em}
\explanation{
	\emph{Example E: Supplementary method}
	\begin{itemize}
		\item The recommended related code generates the {\ttt Layout parameters}, it will be traced by {\ttt getLayoutParam} function, which will further be called inside the query method {\ttt onLayout}. There's a dependency chain between the query and the related code.
	\end{itemize}
}

\\

\bottomrule
\end{tabular}
\end{table*}

\begin{table*}\scriptsize
	\caption{Supplementary code examples}
	\label{tab:diff-examples}
	
	\setlength{\tabcolsep}{0.01\textwidth}
	\begin{tabular}{@{}p{0.49\textwidth}p{0.49\textwidth}@{}}
		\toprule
		Query Code Snippet & Recommended Related Code \\
		\midrule
		

\begin{lstlisting}
public static <K, V extends Comparable<? super V>> SortedSet<Map.Entry<K, V>> 		entriesSortedByValues(Map<K, V> map) {
	SortedSet<Map.Entry<K, V>> sortedEntries = new TreeSet<Map.Entry<K, V>>(
		new Comparator<Map.Entry<K, V>>() {
		@Override
			public int compare(Map.Entry<K, V> e1, Map.Entry<K, V> e2) {
				return e1.getValue().compareTo(e2.getValue());
			}
		});
	sortedEntries.addAll(map.entrySet());
	return sortedEntries;
}
\end{lstlisting}

&
\begin{lstlisting}
public static <K, V extends Comparable<? super V>> Map<K, V> sortByValue( Map<K, V> map ) {
	List<Map.Entry<K, V>> list =
		new LinkedList<Map.Entry<K, V>>( map.entrySet() );
	Collections.sort( list, new Comparator<Map.Entry<K, V>>()
	{
		public int compare( Map.Entry<K, V> o1, Map.Entry<K, V> o2 )
		{
			return (o1.getValue()).compareTo( o2.getValue() );
		}
	} );

	Map<K, V> result = new LinkedHashMap<K, V>();
	for (Map.Entry<K, V> entry : list)
	{
		result.put( entry.getKey(), entry.getValue() );
	}
	return result;
}
\end{lstlisting}
\vspace*{1em}
\explanation{
	\emph{Example F: Different implementation}
	\begin{itemize}
		\item The question title of the SO post is: Sort the values in HashMap.
		\item The query snippet from SO uses {\ttt SortedSet} to store the map entries, while the recommended code provide an alternative, using {\ttt LinkedList}, and show how to use iterate the map.
	\end{itemize}
}


\\

\bottomrule
\end{tabular}
\end{table*}

% Reset listing style
\lstset{
	frame=tb,
    aboveskip=\medskipamount,
    belowskip=\medskipamount,
}

We randomly select 50 SO snippets with its common co-occurring code fragments from the 11,110 groups, and manually examine whether these code fragments are related to the SO input or not, and categorize why we call the relationship a relevant one.

We use $Precision@k$ metric to evaluate the common co-occurring code  which is defined as follows:
\begin{equation}
Precision@k = \frac{1}{N}\sum_{i=1}^{N}\tfrac{\left | relevant_{i,k} \right |}{k}
\end{equation}
where $\left | relevant_{i,k} \right |$ represents the number of positive related results in the top $k$ common co-occurring results for query $i$, $N$ is the number queries we evaluate, which is $50$. $k$ is the number of top results we examine, here we use $k=1$ and $k=3$.

We achieve 80\% and 74.6\% for $Precision@1$ and $Precision@3$ respectively. That is to say, for the 50 most common co-occurring results, 40 of them are manually examined as related, for the 150 top 3 results, 112 of them are related.

We find the following types of relevance in our sample set:
\begin{itemize}
	\item A complementary method that adds more functionality.
	\item A supplementary method that helps with, or gets help from, the query. 
	\item A different implementation for the query.	
\end{itemize}


\begin{table}[h]
	\caption{Categorization of related methods}
	\label{tab:categorization}
	\begin{center}
		\begin{tabular}{ c|c|c } 
			\hline
			Category & Top 1 & Top 3 \\\hline
			Complementary method &  20 (50\%) & 55 (49\%)\\\hline 
			Supplementary method &  18 (45\%) & 53 (47\%) \\ \hline
			Different implementation &  2 (5\%) & 4 (3\%)\\ \hline
			Total related methods & 40 & 112 \\\hline
		\end{tabular}		
	\end{center}

\end{table}
		

\subsubsection{Complementary method} In this category, the query code can function alone, but the related method provides extra functionality to the query code and will further complete the user class. For the example shown as Listing \ref{lst:mot-query} and \ref{lst:mot-related} in Section~\ref{sec:intro}, the query snippet implements unzip a folder in Java.  We find {\ttt zip} function. These two methods can function independently, but often implemented together to get a stronger ability for file manipulation. 

Similarly, we find {\ttt decrypt} function for {\ttt encrypt} and {\ttt onPause} function for {\ttt onCreate} in Table \ref{tab:compl-examples}. The two methods in each pair do not have any direct function call association between them, but they complete each other with extra functionality and are often implemented together in real-life scenarios. 

For the top related methods, half of them are complementary methods. 37\% of the sampled top 3 related methods belong to this category.

\subsubsection{Supplementary method} The related code serves as a helper function to the query, or vice versa. One may make function call to the other. For example the {\ttt merge} function for {\ttt sort}. {\ttt sort} calls {\ttt merge} as a helper function and cannot achieve functionality without it. 

In our first example in Table \ref{tab:suppl-examples}, our related code {\ttt loadDrawable} calls {\ttt queueJob} inside its method body. 
There is another related method being recommended together with {\ttt loadDrawable}, which is shown below in Listings \ref{lst:part2}. {\ttt loadDrawable} also makes a function call to {\ttt getDrawableFromCache} inside its method body, The related methods give the user a broader picture of the whole class, point to a higher level of functionality the user may want to implement, and also direct the user to the most-frequently used higher level functionality and its auxiliaries.

Less than half of the sampled related results are supplementary methods.

\lstset{
	frame=single,
}
\newpage
\begin{lstlisting}[caption={Related method \#2}, label={lst:part2}]
public static Drawable getDrawableFromCache(String url) {
	if (DrawableManager.cache.containsKey(url)) {
		return DrawableManager.cache.get(url);
	}
	
	return null;
}	
\end{lstlisting}


\subsubsection{Different implementation} This category represents those related methods that have similar functionality to the query code. The result provides an alternative, or a more detailed or extended implementation for the functionality. As shown in Table ~\ref{tab:diff-examples}, both of the methods implement sorting values in a {\ttt Map}, the query store the map entries in a {\ttt SortedSet}, while the related code uses {\ttt LinkedList}, and shows how to iterate a {\ttt Map}. For the {\ttt encrypt} function in Table ~\ref{tab:compl-examples}, the related code also provide an alternative implementation with {\ttt String} inputs, as shown in Listing ~\ref{lst:encryt}.

A small number of sampled related methods provide different implementation to the query itself.


\begin{lstlisting}[caption={different implementation for \texttt{encrypt}}, label={lst:encryt}]
public static String encrypt(final String password, String message) throws GeneralSecurityException {
	try {
		final SecretKeySpec key = generateKey(password);
		log("message", message);
		byte[] cipherText = encrypt(key, ivBytes, message.getBytes(CHARSET));
		//NO_WRAP is important as was getting \n at the end
		String encoded = String.valueOf(
			Base64.encodeToString(cipherText, Base64.NO_PADDING ));
		log("Base64.NO_WRAP", encoded);
		return encoded;
	} catch (UnsupportedEncodingException e) {
		if (DEBUG_LOG_ENABLED)
			Log.e(TAG, "UnsupportedEncodingException ", e);
		throw new GeneralSecurityException(e);
	}
}
\end{lstlisting}


From the in-depth manual analysis from this section, we can see that there is a large amount of related code among the common co-occurring code, and they are worth to be considered for recommendation besides similar code to the query.



\section{Related Work}
\label{sec:related}
Various code search techniques have been proposed to discover relevant code components (e.g., functions, code snippets) given a user query. For example, given a keyword query, Portfolio retrieves function definitions and their usages using a combination of a PageRank model and an association model~\cite{mcmillan2011portfolio}. Chan et al. improve Portfolio by matching the textual similarity between containing nodes in an API usage subgraph with a keyword query~\cite{chan2012searching}. CodeHow also finds code snippets relevant to a natural language query. It explores API documents to identify relationships between query terms and APIs~\cite{lv2015codehow}. Instead of using natural language queries, several techniques automatically recommend relevant code snippets based on contextual information such as types in a target program~\cite{Holmes2005, sahavechaphan2006xsnippet, thummalapenta2007parseweb, ponzanelli2014mining}. CodeGenie is a test-driven code search technique that allows developers to specify desired functionality via test cases and then matches relevant methods and classes with the given test~\cite{lazzarini2009applying}. To more precisely capture search intent, S6 allows developers to express desired functionality using a combination of input-output types, test cases, and keyword descriptions~\cite{reiss2009semantics}.

Code-to-code search tools are most related to our technique among all different kinds of code search techniques. Given a code snippet as input, FaCoY~\cite{kim2018Facoy} finds semantically similar code snippets in a Stack Overflow dataset by first matching with accompanied natural language descriptions in related posts instead of matching code directly. Unlike FaCoY, several techniques infer an underlying code search pattern from a given code fragment~\cite{zhang2015interactive, MKM:11, meng2013lase, sivaraman2019active}. Sydit generalizes concrete identifiers (e.g., variable names, types, and method calls) in a given code example as an abstract code template and identifies other similar locations via AST-based tree matching~\cite{MKM:11}. Lase uses multiple code examples instead of a single example to better infer the search intent of a user~\cite{meng2013lase}. Critics allows developers to construct an AST-based search pattern from a single example through manual code selection, customization, and parameterization~\cite{zhang2015interactive}. These code-to-code search tools focus on identifying relevant code snippets that are syntactically or semantically similar to a given code snippet. However, since many programming tasks (e.g., password encryption and decryption) require multiple code snippets or functions to work together, none of the existing techniques recommend code snippets that complement a given code snippet to complete desired functionality.



\section{Conclusion}
\label{sec:conclude}





%\section*{References}

\bibliographystyle{plain}

\bibliography{ref}

\end{document}
