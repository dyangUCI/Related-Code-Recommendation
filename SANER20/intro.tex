\section{Introduction}
\label{sec:intro}
%\todo{color the code examples based on Java syntax. Be careful with the use of "related methods", "recommended methods", "common co-occurring methods"}
Over the past decade, code search has emerged as an interesting, but
challenging, topic to both industry and research communities. Various code search
techniques have been proposed in the
literature~\cite{bajracharya2009sourcerer,reiss2009semantics,lazzarini2009applying,mcmillan2012exemplar}, and
some search code engines have been implemented and are, or were, publicly
available~\cite{googlesearch, github,codase,krugle,ohloh,searchcode}. Code search engines 
take some specification as input (e.g., a keyword description, a code fragment, or a test) and 
recommends pieces of code that match the given specification based on some forms of ``similarity'' measurement. Prior work has shown that by identifying and contrasting similar code, programmers could quickly understand the gist of implementing a function and explore potential variations to write more complete and robust code~\cite{zhang2019analyzing, luan2019aroma}. For instance, by inspecting variations in similar code found in GitHub, programmers are able to identify critical code parts such as safety checks and exception handling logics that are missed in a given code example~\cite{zhang2019analyzing}.

% motivation example

This work explores the opportunities of searching relevant code beyond similarity. Consider
the following scenario. A programmer is implementing a Java method for
file decompression. A code search engine may recommend a code example as as shown in 
Listing~\ref{lst:mot-query} in Figure~\ref{fig:example}, which takes the path to a zip file as input and
unpacks all files within the zip into a target directory.  This piece of code is sufficient for 
a simple program task of unpacking a zip file. However, in practice, the programmer 
may undertake a more complex programming task where unzipping a file is a small,
integral part. Therefore, the programmer may also want to know what else may be
related to this functionality. 
%The programmer may also want to have a comprehensive understanding of the collection of related
%functions for zip file manipulation, in general.
 If an additional functionality often co-occurs with unzipping, the programmer may want to add it
to her own project as needed. 
For instance, Listing \ref{lst:mot-related} shows an example of this kind of additional 
functionalities---a method that zips a list of files from a folder into the target zip file. Unzipping
and zipping are two kinds of file manipulation in the opposite direction. 
Though these two functions work independently, they are often implemented together in a code base to complement each other. We consider the zip method and the unzip method {\em related} to each other, or {\em complementary code fragment}, to be more specifically.

\begin{figure*}[!t]
\begin{minipage}[t]{0.5\linewidth}
\begin{lstlisting}[style=MyJavaSmallStyle, caption={Query code\todo{Simply listing 1 to a similar length as listing 2}}, label={lst:mot-query}]
public static boolean unpackZip(String path, String zipname, String targetDirectory) {
	InputStream is;
	ZipInputStream zis;
	try {
		String filename;
		is = new FileInputStream(path + zipname);
		zis = new ZipInputStream(new BufferedInputStream(is));
		ZipEntry ze;
		byte[] buffer = new byte[1024];
		int count;

		while ((ze = zis.getNextEntry()) != null) {
			filename = ze.getName();

			if (ze.isDirectory()) {
				File fmd = new File(targetDirectory + filename);
				fmd.mkdirs();
				continue;
			}

			FileOutputStream fout = new FileOutputStream(targetDirectory + filename);

			while ((count = zis.read(buffer)) != -1) {
				fout.write(buffer, 0, count);
			}

			fout.close();
			zis.closeEntry();
		}

		zis.close();
	} catch (IOException e) {
		e.printStackTrace();
		return false;
	}

	return true;
}
\end{lstlisting}
\end{minipage}
%
\begin{minipage}[t]{0.5\linewidth}
\begin{lstlisting}[style=MyJavaSmallStyle, caption={Recommended related code}, label={lst:mot-related}]
public static void zip(String baseFolder, List<File> files, String zipFile) {
	try  {
		BufferedInputStream origin = null;
		FileOutputStream dest = new FileOutputStream(zipFile);

		ZipOutputStream out = new ZipOutputStream(new BufferedOutputStream(dest));
		byte data[] = new byte[BUFFER];

		for (File file : files) {
			FileInputStream fi = new FileInputStream(file);
			origin = new BufferedInputStream(fi, BUFFER);
			String relativeFileName = file.getAbsolutePath().replace(baseFolder + File.separator , """");
			ZipEntry entry = new ZipEntry(relativeFileName);
			out.putNextEntry(entry);
			int count;
			while ((count = origin.read(data, 0, BUFFER)) != -1) {
				out.write(data, 0, count);
			}
			origin.close();
		}

		out.close();
	} catch(Exception e) {
		e.printStackTrace();
	}

}
\end{lstlisting}
\end{minipage}
\end{figure*}
% code search and code completion
 
Code-to-code search engines could be leveraged to identify related code given a code 
fragment of interest~\cite{kim2018Facoy, krugle, searchcode}. 
However, these techniques find syntactically or semantically similar code fragments 
only, without considering about auxiliary or complementary functionality. 
For instance, given an unzip function, they cannot find a complementary zip function, since neither
the implementation nor the functionality of these two operations are similar. 
Currently,  there is also limited understanding about what other kinds of relevant code beyond similar code
may exist in practice and thus should be recommended modern code search engines. 

%Pattern-based code completion tools~\cite{nguyen2009groum,
%  nguyen2012grapacc} also recommend completing code for a given code
%query. They do so by mining common API usage patterns from a large
%code corpus. For a given partial snippet as query, if it matches a
%prefix of a mined pattern, the tool recommends the rest of the pattern
%for completion. Again, such tools only work for the mined patterns;
%that is, they do not recommend code outside the mined patterns.

%For both code search engines and pattern-based code completion tools,
%the retrieved code snippets may have extra lines of code with more
%functionality, but they are not designed to search for commonly used
%additional code.

Complementary methods shown in Figure~\ref{fig:example} often co-occur in the same source file or the same code base, which serves as an interesting property to exploit for recommending relevant code examples.
Manually identifying co-occurring code is tedious and time-consuming, 
since some co-occurring code may be project-specific and thus not relevant to a given code query. 
Therefore, we first build an automated approach to identify, cluster, and rank common, 
co-occurring code given a code query.  
Our approach uses a state-of-the-art clone detecor called SourcererCC~\cite{sajnani2016sourcerercc} to identify similar
counterparts (i.e., clones) in a large code corpus. Then our approach contrasts surrounding code of those clones 
and identify common code that are shared around mutiple clones.

Using 21K Java code snippets from Stack Overflow
as code queries, we automatically identify relevant code of these code queries in a large corpus of  50K GitHub projects with at least five stars. 
As a result, we obtained 21K groups of similar code in GitHub. We manually inspected a random sample of 50 common, co-occurring code fragments and examined their relevancy to the original query. 
74\% of those common, co-occurring code fragments represented relevant functionality, 
which should be included in code search results. 
Furthermore, we identified three major types of relevant co-occurring code---{\em complementary}, {\em supplementary}, and {\em alternative} functions. These findings show that it is beneficial to recommend common, co-occurring code of a given code query to achieve more complete functionalities, instead of just recommending similar code. 

To further demonstrate this idea, we implement a Chrome extension called {\tool} that recommends relevant but non-similar code when programmers browse code examples in Stack Overflow. It is well known that programmers often search and reuse online examples during modern software development~\cite{brandt2010example, umarji2008archetypal, gallardo2009internet}. {\tool} augments this programming workflow by reminding programmers what other complementary, supplementary, or alternative functions should also be included as they copy and paste code from the Web. We compared {\tool} with a state-of-the-art code search engine called Facoy~\cite{kim2018Facoy}. Among ten sample search queries, FaCoy only identified related code for one query since the related code is similar to the code query. A general-purpose search engine, Google Search, was able to identify GitHub files that contain related code recommended by {\tool}  for half of the queries. However, Google could only retrieve full files where programmers still had to manually go through those files to identify related code. By contrast, {\tool} pinpointed where related code fragments were based on how frequently they occurred in other similar locations.
%Given a SO example, {\tool} automatically detects similar code fragments in GitHub and retrieves common code that co-occurs with these fragments. {\tool} also incorporates a clustering algorithm on top of co-occurrence to provide ranking. 
%Though {\tool} currently supports Java, it is not restricted to any programming languages. The Java parser is only used to tokenize code fragments for clone detection and can be substituted with any off-the-shelf parsers of other languages.
%The goal of our work is to explore the search needs shown in Listings
%\ref{lst:mot-query} and \ref{lst:mot-related}. In this paper, we first
%describe {\tool}, an approach for retrieving common co-occurring code fragments. 
%Given a code snippet as input query and a large corpus of code containing
%millions of code fragments, {\tool} returns a set of code
%fragments such that:
%\begin{itemize}
%	\item the code fragments co-occur with similar counterparts of the input query.
%	\item the code fragments are ranked by their commonality.
%\end{itemize}


%For the time being, we focus on method-level code fragments written in
%Java. Both the query snippet and the common co-occurring code
%fragments are Java methods. {\tool} works by first tokenizing the
%query and all methods in the code corpus. It then uses token
%similarity to detect similar counterparts to the query in the code
%corpus. We delegate this process to a clone detection tool,
%SourcererCC~\cite{sajnani2016sourcerercc}. Finally, {\tool} recognizes
%other methods which co-occur with these similar counterparts as
%candidate related methods.

% advantages of CodeAid
% {\tool} has the following properties that differentiates it from traditional similarity based code search engines:
%\begin{itemize}
%	\item It retrieves functionality groups, which the user may want to implement together with the input query.
%	\item It has a clustering algorithm on top of co-occurrence to provide ranking.
%	\item It is not restricted to any programming language. The Java parser is only used to chunk the file into methods, it can be replaced by the parser from any languages as needed.
%	\item It has flexible granularity level. We can chunk the files into blocks of any size. All similarity comparison processes are token-based, which means as long as we have the token list representing the block, it does not matter what size the block is.
%	\item It is fast enough to be used in real time. The most
%          time-consuming part is similar code detection. However,
%          generating the indexes is a one-time task and can be done
%          before any query is processed.	
%\end{itemize}

%We apply our approach on two data sets: (1) a query dataset
%consisting of 21,207 Java code snippets collected from Stack Overflow
%(SO), and (2) a large Java code base consisting of 50,826 projects which 
%have at least five starts collected from GitHub, with over 5.8M distinct 
%Java files. We then run each query in the query dataset through {\tool} 
%and collect the common co-occurring fragments coming from the code base. 
%We manually inspected a random sample of 50 commonly co-occurring code fragments and examined their relevancy to the original query. 
%We found 76\% of them represent relevant functionality that should be included in code search results.
%Furthermore, we identified three major types of relevant co-occurring code---{\em complementary}, {\em supplementary}, and {\em alternative} functions.
%We build a Chrome extension for Stack Overflow to help with the qualitative evaluation and also to provide one application of recommending relative code beyond similarity.

%\todo{summarize the contributions}
In summary, this paper makes the following contribution:
\begin{itemize}
\item We present a new code search method that recommends common, co-occuring code of a given code query, rather than only recommending similar code.
\item We empirically show the prevalence of common, co-occurring code by quantifying the commonality of sorrounding code of GitHub clones. We also find that the majority of such co-occurring code fragments represent meanful functionality such as complementary, supplementary, or alternative functions, which should be recommended by modern code search engines.
\item We develop a Chrome extension called {\tool} to recommend related code during online code search and demonstrate that {\tool} is capable of recommending related code that cannot be identified by a state-of-the-art code search tool.
%\item It retrieves a large dataset of commonly co-occurring code.
%\item It provides a in-depth relevance analysis between the commonly co-occurring code with its original query. These code proved to be related and code should be considered for recommendation along with the similar ones.
%\item It implements a chrome extension as a tool for recommending related code in practice.
\end{itemize}

The rest of the paper is organized as follows: Section
\ref{sec:approach} describes the approach for generating
common co-occurring methods, with ranking. Section \ref{sec:eval} presents the manual analysis result of common co-occurring code in terms of its relevance to the original query. We also explore whether recommended code from  existing search engines can provide same relevance. \ref{sec:chrome} illustrates the instantiation of {\tool} as a chrome extension as well as a use scenario in a web browser. Section \ref{sec:related} describes and Section \ref{sec:conclude} concludes the paper.
