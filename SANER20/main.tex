\documentclass[conference]{IEEEtran}
\IEEEoverridecommandlockouts
% The preceding line is only needed to identify funding in the first footnote. If that is unneeded, please comment it out.
\usepackage{cite}
\usepackage{amsmath,amssymb,amsfonts}
\usepackage{algorithmic}
\usepackage{graphicx}
\usepackage{textcomp}
\usepackage{xcolor}
\def\BibTeX{{\rm B\kern-.05em{\sc i\kern-.025em b}\kern-.08em
    T\kern-.1667em\lower.7ex\hbox{E}\kern-.125emX}}
\begin{document}

\title{Are co-occurring code fragments useful for recommendation?\\
%{\footnotesize \textsuperscript{*}Note: Sub-titles are not captured in Xplore and
%should not be used}
%\thanks{Identify applicable funding agency here. If none, delete this.}
}

%\author{\IEEEauthorblockN{1\textsuperscript{st} Given Name Surname}
%\IEEEauthorblockA{\textit{dept. name of organization (of Aff.)} \\
%\textit{name of organization (of Aff.)}\\
%City, Country \\
%email address}
%\and
%\IEEEauthorblockN{2\textsuperscript{nd} Given Name Surname}
%\IEEEauthorblockA{\textit{dept. name of organization (of Aff.)} \\
%\textit{name of organization (of Aff.)}\\
%City, Country \\
%email address}
%\and
%\IEEEauthorblockN{3\textsuperscript{rd} Given Name Surname}
%\IEEEauthorblockA{\textit{dept. name of organization (of Aff.)} \\
%\textit{name of organization (of Aff.)}\\
%City, Country \\
%email address}
%\and
%\IEEEauthorblockN{4\textsuperscript{th} Given Name Surname}
%\IEEEauthorblockA{\textit{dept. name of organization (of Aff.)} \\
%\textit{name of organization (of Aff.)}\\
%City, Country \\
%email address}
%\and
%\IEEEauthorblockN{5\textsuperscript{th} Given Name Surname}
%\IEEEauthorblockA{\textit{dept. name of organization (of Aff.)} \\
%\textit{name of organization (of Aff.)}\\
%City, Country \\
%email address}
%\and
%\IEEEauthorblockN{6\textsuperscript{th} Given Name Surname}
%\IEEEauthorblockA{\textit{dept. name of organization (of Aff.)} \\
%\textit{name of organization (of Aff.)}\\
%City, Country \\
%email address}
%}

\maketitle

\begin{abstract}
Recommending similar code for a given code query can help programmers detect
situations that they had overlooked, or that they did not know how to solve.
Most code-to-code search tools aim at finding syntactically or semantically
similar code given some code of interest. These tools focus on the counterparts
themselves, while ignoring the co-occurring code fragments with the counterpart.
Such co-occurring code fragments are not semantically similar to the code query,
but could be important to work together with the given code to accomplish a
complete or related functionality. In this paper, we dig deeper into the
relationship between the co-occurring code fragments and the original query, and
explore the possibility of recommending useful co-occurring code to the user.

In this paper, we describe a technique to find and rank co-occurring code
fragments to a given code query in a code corpus. Specifically, we use GitHub
Java projects as our corpus and focus on method-level code retrieval. Given a
code fragment of interest, we use a code clone detection tool to detect all its
similar counterparts across different GitHub files, and then rank other methods
in the files based on clustering statistics. We keep the top ranked co-occurring
methods as related ones.

In order to evaluate the related code fragments, we use Stack Overflow (SO) code
snippets as queries, and measure the precision of the retrieved related methods.
We gather a query code base of 21,207 SO code snippets written in Java. We
discovered related co-occurring code fragments in GitHub for 11,110 of these
snippets. We also perform an in-depth analysis on a sample of these related code
snippets and categorize the relationship between a SO query snippet and related
code, sheding light on how related code can be useful in practice. We then
manually evaluate the usefulness of related GitHub code, and get a precision of
75.6\%. Our results show that top-ranked co-occurring code fragments are useful 
for the users, and we provide a Chrome extension as one possible implementation
of recommending related co-occurring code fragments in real life.
\end{abstract}

\begin{IEEEkeywords}
code recommendation, related code, code search, mining software repositories
\end{IEEEkeywords}

\section{Introduction}
This document is a model and instructions for \LaTeX.
Please observe the conference page limits. 





\end{document}
