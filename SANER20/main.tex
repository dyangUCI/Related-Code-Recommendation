\documentclass[conference]{IEEEtran}
\IEEEoverridecommandlockouts
% The preceding line is only needed to identify funding in the first footnote. If that is unneeded, please comment it out.
\usepackage{cite}
\usepackage{amsmath,amssymb,amsfonts}
\usepackage{algorithmic}
\usepackage{graphicx}
\usepackage{textcomp}
\usepackage{xcolor}

\newcommand\todo[1]{\textcolor{red}{TODO: #1}}


\def\BibTeX{{\rm B\kern-.05em{\sc i\kern-.025em b}\kern-.08em
    T\kern-.1667em\lower.7ex\hbox{E}\kern-.125emX}}
\begin{document}

%\title{Are co-occurring code fragments useful for code recommendation?}
\title{Towards Searching Beyond Similar Code}
%{\footnotesize \textsuperscript{*}Note: Sub-titles are not captured in Xplore and
%should not be used}
%\thanks{Identify applicable funding agency here. If none, delete this.}


%\author{\IEEEauthorblockN{1\textsuperscript{st} Given Name Surname}
%\IEEEauthorblockA{\textit{dept. name of organization (of Aff.)} \\
%\textit{name of organization (of Aff.)}\\
%City, Country \\
%email address}
%\and
%\IEEEauthorblockN{2\textsuperscript{nd} Given Name Surname}
%\IEEEauthorblockA{\textit{dept. name of organization (of Aff.)} \\
%\textit{name of organization (of Aff.)}\\
%City, Country \\
%email address}
%\and
%\IEEEauthorblockN{3\textsuperscript{rd} Given Name Surname}
%\IEEEauthorblockA{\textit{dept. name of organization (of Aff.)} \\
%\textit{name of organization (of Aff.)}\\
%City, Country \\
%email address}
%\and
%\IEEEauthorblockN{4\textsuperscript{th} Given Name Surname}
%\IEEEauthorblockA{\textit{dept. name of organization (of Aff.)} \\
%\textit{name of organization (of Aff.)}\\
%City, Country \\
%email address}
%\and
%\IEEEauthorblockN{5\textsuperscript{th} Given Name Surname}
%\IEEEauthorblockA{\textit{dept. name of organization (of Aff.)} \\
%\textit{name of organization (of Aff.)}\\
%City, Country \\
%email address}
%\and
%\IEEEauthorblockN{6\textsuperscript{th} Given Name Surname}
%\IEEEauthorblockA{\textit{dept. name of organization (of Aff.)} \\
%\textit{name of organization (of Aff.)}\\
%City, Country \\
%email address}
%}

\maketitle

\begin{abstract}
%Recommending related code of a given query can help programmers detect
%situations that they have overlooked, or that they did not know how to solve.
Code-to-code search can help developers find similar solutions and identify possible improvment opportunities or alternative choices. 
%Existing code-to-code search tools only identify syntactically or semantically
%similar code given a code of interest. These tools focus on the counterparts
%themselves, while ignoring the co-occurring code fragments with the counterpart.
Existing search tools only recommend other code locations that are syntactically similar to the given code but do not reason about other kinds of relevant code that a developer should also pay attention to, e.g., an auxiliary function to accomplish a complete task.  
%Though such surrouding, co-occurring code fragments are not similar to the given code query, they might be necessary to work together with the given code to accomplish a complete task or provide auxilary functionality. 
There is also limited understanding about what other kinds of code relevancy may exist in practice beyond code similarity and thus should be incorporated into modern code search engines. 
To bridge the gap, this paper presents an in-depth analysis of other code fragments that co-occur with similar code that a regular search engine typically recommends and their relevance to the given code query.

To this end, we first constructed a large dataset of 21K groups of similar code written in Java, using Stack Overflow code snippets as queries and identifying their counterparts in GitHub via clone detection. For more than half of these SO code snippets, their GitHub counterparts share common code that co-occurs in the same Java file yet not similar to the original queries.\todo{Di: let's do another experiment to report the similarity of those common related code fragments to the original queries.} We manually inspected a random sample of \todo{X} commonly co-occurring code fragments and found 76\% of them represent relevant functionality that should be included in code search results. Furthermore, we identified three major types of relevant co-occurring code---{\em complementary}, {\em supplementary}, and {\em alternative} functions. This study result calls for a new search engine that accounts for such code relevance beyond code similarity.  %the prevalence and usefulness of relevant co-occuring functions. At the end, we implemented a code search tool that incorporates such relevant functions as a possible tool design.

%In this paper, we describe a technique to find and rank co-occurring code
%fragments to a given code query in a code corpus. Specifically, we use GitHub
%Java projects as our corpus and focus on method-level code retrieval. Given a
%code fragment of interest, we use a code clone detection tool to detect all its
%similar counterparts across different GitHub files, and then rank other methods
%in the files based on clustering statistics. We keep the top ranked co-occurring
%methods as related ones.

%In order to evaluate the related code fragments, we use Stack Overflow (SO) code
%snippets as queries, and measure the precision of the retrieved related methods.
%We gather a query code base of 21,207 SO code snippets written in Java. We
%discovered related co-occurring code fragments in GitHub for 11,110 of these
%snippets. We also perform an in-depth analysis on a sample of these related code
%snippets and categorize the relationship between a SO query snippet and related
%code, sheding light on how related code can be useful in practice. We then
%manually evaluate the usefulness of related GitHub code, and get a precision of
%75.6\%. Our results show that top-ranked co-occurring code fragments are useful 
%for the users, and we provide a Chrome extension as one possible implementation
%of recommending related co-occurring code fragments in real life.
\end{abstract}

\begin{IEEEkeywords}
code recommendation, related code, code search, mining software repositories
\end{IEEEkeywords}

\section{Introduction}
This document is a model and instructions for \LaTeX.
Please observe the conference page limits. 





\end{document}
