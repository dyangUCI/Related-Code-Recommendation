%\section{Evaluation}
%\label{sec:eval}
%
%
%
%In this section, we describe the design of the assessment scenarios for \tool\ and report the evaluation results. Specifically, our experiments aim to address the following research questions:
%\begin{itemize}
%	\item \textbf{RQ1}: Can {\tool} retrieve related code fragments for the queries in query code base?
%	\item \textbf{RQ2}: Are the code fragments recommended by \tool\ related to the query?
%	\item \textbf{RQ3}: What kinds of related code fragments do \tool\ recommend?
%	\item \textbf{RQ4}: Can we get the recommended related code fragments from code search engines?
%\end{itemize}
%

\section{Manual analysis and categorization}
\label{sec:eval}
% Set listing style for table
\lstset{
	frame=none,
    aboveskip=0pt,
    belowskip=0pt,
    basicstyle=\tiny\ttfamily,
}
\begin{table*}\scriptsize
\caption{Complementary code examples}
\label{tab:compl-examples}

\setlength{\tabcolsep}{0.01\textwidth}
\begin{tabular}{@{}p{0.49\textwidth}p{0.49\textwidth}@{}}
\toprule
Query Code Snippet & Recommended Related Code \\
\midrule



\begin{lstlisting}
public static boolean unpackZip(String path, String zipname, String targetDirectory) {
	InputStream is;
	ZipInputStream zis;
	try {
		String filename;
		is = new FileInputStream(path + zipname);
		zis = new ZipInputStream(new BufferedInputStream(is));
		ZipEntry ze;
		byte[] buffer = new byte[1024];
		int count;

		while ((ze = zis.getNextEntry()) != null) {
			filename = ze.getName();

			if (ze.isDirectory()) {
				File fmd = new File(targetDirectory + filename);
				fmd.mkdirs();
				continue;
			}

			FileOutputStream fout = new FileOutputStream(targetDirectory + filename);

			while ((count = zis.read(buffer)) != -1) {
				fout.write(buffer, 0, count);
			}

			fout.close();
			zis.closeEntry();
		}

		zis.close();
	} catch (IOException e) {
		e.printStackTrace();
		return false;
	}

	return true;
}
\end{lstlisting}


&
\begin{lstlisting}
public static void zip(String baseFolder, List<File> files, String zipFile) {
	try  {
		BufferedInputStream origin = null;
		FileOutputStream dest = new FileOutputStream(zipFile);

		ZipOutputStream out = new ZipOutputStream(new BufferedOutputStream(dest));
		byte data[] = new byte[BUFFER];

		for (File file : files) {
			FileInputStream fi = new FileInputStream(file);
			origin = new BufferedInputStream(fi, BUFFER);
			String relativeFileName = file.getAbsolutePath().replace(baseFolder + File.separator , """");
			ZipEntry entry = new ZipEntry(relativeFileName);
			out.putNextEntry(entry);
			int count;
			while ((count = origin.read(data, 0, BUFFER)) != -1) {
				out.write(data, 0, count);
			}
			origin.close();
		}

		out.close();
	} catch(Exception e) {
		e.printStackTrace();
	}

}
\end{lstlisting}
\vspace*{1em}
\explanation{
    \emph{Example A: Complementary method}
    \begin{itemize}
        \item The query snippet implements unzip a folder in Java.
        \item The recommended related method implements zip a folder. These two methods can function independently, but often implemented together to get a stronger ability for folder manipulation.
    \end{itemize}
}

\\

\bottomrule

\begin{lstlisting}
public static byte[] encrypt(final SecretKeySpec key, final byte[] iv, final byte[] message)
throws GeneralSecurityException {
	final Cipher cipher = Cipher.getInstance(AES_MODE);
	IvParameterSpec ivSpec = new IvParameterSpec(iv);
	cipher.init(Cipher.ENCRYPT_MODE, key, ivSpec);
	byte[] cipherText = cipher.doFinal(message);

	log(""cipherText"", cipherText);

	return cipherText;
}
\end{lstlisting}


&
\begin{lstlisting}
public static byte[] decrypt(final SecretKeySpec key, final byte[] iv, final byte[] decodedCipherText)
throws GeneralSecurityException {
	final Cipher cipher = Cipher.getInstance(AES_MODE);
	IvParameterSpec ivSpec = new IvParameterSpec(iv);
	cipher.init(Cipher.DECRYPT_MODE, key, ivSpec);
	byte[] decryptedBytes = cipher.doFinal(decodedCipherText);

	log(""decryptedBytes"", decryptedBytes);

	return decryptedBytes;
}
\end{lstlisting}

\vspace*{1em}
\explanation{
	\emph{Example B: Complementary method}
	\begin{itemize}
		\item The query snippet implements \texttt{encrypt} functionality for an byte array.
		\item The recommended related method decrypts a decoded byte array. 
	\end{itemize}
}

\\

\bottomrule

\begin{lstlisting}
@Override
public void onCreate(Bundle savedInstanceState) {
	super.onCreate(savedInstanceState);
	setContentView(R.layout.main);
	preferred = (TextView)findViewById(R.id.preferred);
	orientation = (TextView)findViewById(R.id.orientation);
	mgr = (SensorManager) this.getSystemService(SENSOR_SERVICE);
	accel = mgr.getDefaultSensor(Sensor.TYPE_ACCELEROMETER);
	compass = mgr.getDefaultSensor(Sensor.TYPE_MAGNETIC_FIELD);
	orient = mgr.getDefaultSensor(Sensor.TYPE_ORIENTATION);
	WindowManager window = (WindowManager)
	this.getSystemService(WINDOW_SERVICE);
	int apiLevel = Integer.parseInt(Build.VERSION.SDK);
	if(apiLevel <8) {
		mRotation = window.getDefaultDisplay().getOrientation();
	}
	else {
		mRotation = window.getDefaultDisplay().getRotation();
	}
	}
\end{lstlisting}

&
\begin{lstlisting}
@Override
protected void onPause() {
	mgr.unregisterListener(this, accel);
	mgr.unregisterListener(this, compass);
	mgr.unregisterListener(this, orient);
	super.onPause();
}
\end{lstlisting}

\vspace*{1em}
\explanation{
	\emph{Example C: Complementary method}
	\begin{itemize}
		\item The query snippet implements \texttt{onCreate} functionality for an \texttt{Android} activity.
		\item The recommended related method implements \texttt{onPause} which does not have direct function call with the query snippet, but adds extra functionality to the activity. 
	\end{itemize}
}

\\

\bottomrule
\end{tabular}
\end{table*}

\begin{table*}\scriptsize
	\caption{Supplementary code examples}
	\label{tab:suppl-examples}
	
	\setlength{\tabcolsep}{0.01\textwidth}
	\begin{tabular}{@{}p{0.49\textwidth}p{0.49\textwidth}@{}}
		\toprule
		Query Code Snippet & Recommended Related Code \\
		\midrule
		

\begin{lstlisting}
private void queueJob(final String url, final ImageView imageView,final Drawable placeholder) {
	/* Create handler in UI thread. */
	final Handler handler = new Handler() {
	@Override
	public void handleMessage(Message msg) {
		String tag = mImageViews.get(imageView);
			if (tag != null && tag.equals(url)) {
				if (imageView.isShown())
					if (msg.obj != null) {
						imageView.setImageDrawable((Drawable) msg.obj);
					} else {
					imageView.setImageDrawable(placeholder);
					//Log.d(null, "fail " + url);
					}
				}
			}
	};

	mThreadPool.submit(new Runnable() {
		@Override
		public void run() {
			final Drawable bmp = downloadDrawable(url);
			// if the view is not visible anymore, the image will be ready for next time in cache
			if (imageView.isShown())
			{
				Message message = Message.obtain();
				message.obj = bmp;
				//Log.d(null, "Item downloaded: " + url);

				handler.sendMessage(message);
			}
		}
	});
}
\end{lstlisting}


&


\begin{lstlisting}
public void loadDrawable(final String url, final ImageView imageView) {
	imageViews.put(imageView, url);
	Drawable drawable = getDrawableFromCache(url);
	// check in UI thread, so no concurrency issues
	if (drawable != null) {
		Log.d(null, "Item loaded from cache: " + url);
		imageView.setImageDrawable(drawable);
	} else {
		imageView.setImageDrawable(placeholder);
		queueJob(url, imageView);
	}
}
\end{lstlisting}

\vspace*{1em}
\explanation{
	\emph{Example D: Supplementary method}
	\begin{itemize}
		\item The recommended related method calls the query snippet within its method body. It is a higher-level funtionality to the query.
	\end{itemize}
}

\\


\bottomrule


\begin{lstlisting}
@Override
protected void onLayout(boolean changed, int l, int t, int r, int b) {
	final int count = getChildCount();
	for (int i = 0; i < count; i++) {
		View child = getChildAt(i);
		LayoutParams lp = (LayoutParams) child.getLayoutParams();
		child.layout(lp.x+5, lp.y+5, lp.x + child.getMeasuredWidth(), lp.y + child.getMeasuredHeight());
	}
}
\end{lstlisting}

&
\begin{lstlisting}
@Override
protected LayoutParams generateLayoutParams(ViewGroup.LayoutParams p) {
	return new LayoutParams(p);
}
\end{lstlisting}

\vspace*{1em}
\explanation{
	\emph{Example E: Supplementary method}
	\begin{itemize}
		\item The recommended related code generates the \\texttt{Layout parameters}, it will be traced by \\texttt{getLayoutParam} function, which will further be called inside the query method \\texttt{onLayout}. There's a dependency chain between the query and the related code.
	\end{itemize}
}

\\

\bottomrule
\end{tabular}
\end{table*}

\begin{table*}\scriptsize
	\caption{Supplementary code examples}
	\label{tab:diff-examples}
	
	\setlength{\tabcolsep}{0.01\textwidth}
	\begin{tabular}{@{}p{0.49\textwidth}p{0.49\textwidth}@{}}
		\toprule
		Query Code Snippet & Recommended Related Code \\
		\midrule
		

\begin{lstlisting}
public static <K, V extends Comparable<? super V>> SortedSet<Map.Entry<K, V>> 		entriesSortedByValues(Map<K, V> map) {
	SortedSet<Map.Entry<K, V>> sortedEntries = new TreeSet<Map.Entry<K, V>>(
		new Comparator<Map.Entry<K, V>>() {
		@Override
			public int compare(Map.Entry<K, V> e1, Map.Entry<K, V> e2) {
				return e1.getValue().compareTo(e2.getValue());
			}
		});
	sortedEntries.addAll(map.entrySet());
	return sortedEntries;
}
\end{lstlisting}

&
\begin{lstlisting}
public static <K, V extends Comparable<? super V>> Map<K, V> sortByValue( Map<K, V> map ) {
	List<Map.Entry<K, V>> list =
		new LinkedList<Map.Entry<K, V>>( map.entrySet() );
	Collections.sort( list, new Comparator<Map.Entry<K, V>>()
	{
		public int compare( Map.Entry<K, V> o1, Map.Entry<K, V> o2 )
		{
			return (o1.getValue()).compareTo( o2.getValue() );
		}
	} );

	Map<K, V> result = new LinkedHashMap<K, V>();
	for (Map.Entry<K, V> entry : list)
	{
		result.put( entry.getKey(), entry.getValue() );
	}
	return result;
}
\end{lstlisting}
\vspace*{1em}
\explanation{
	\emph{Example F: Different implementation}
	\begin{itemize}
		\item The question title of the SO post is: Sort the values in HashMap.
		\item The query snippet from SO uses \texttt{SortedSet} to store the map entries, while the recommended code provide an alternative, using \texttt{LinkedList}, and show how to use iterate the map.
	\end{itemize}
}


\\

\bottomrule
\end{tabular}
\end{table*}

% Reset listing style
\lstset{
	frame=tb,
    aboveskip=\medskipamount,
    belowskip=\medskipamount,
}

We randomly select 50 SO snippets with its commonly co-occurring code fragments from the 11,110 groups, and manually examine whether these code fragments are related to the SO input or not, and categorize why we call the relationship a relevant one.

We use $Precision@k$ metric to evaluate the commonly co-occurring code  which is defined as follows:
\begin{equation}
Precision@k = \frac{1}{N}\sum_{i=1}^{N}\tfrac{\left | relevant_{i,k} \right |}{k}
\end{equation}
where $\left | relevant_{i,k} \right |$ represents the number of positive related results in the top $k$ commonly co-occurring results for query $i$, $N$ is the number queries we evaluate, which is $50$. $k$ is the number of top results we examine, here we use $k=1$ and $k=3$.

We achieve 80\% and 74.6\% for $Precision@1$ and $Precision@3$ respectively. That is to say, for the 50 top 1 commonly co-occurring results, 40 of them are manually examined as related, for the 150 top 3 results, 112 of them are related.

We find the following types of relevance in our sample set:
\begin{itemize}
	\item A complementary method that adds more functionality.
	\item A supplementary method that helps with, or gets help from, the query. 
	\item A different implementation for the query.	
\end{itemize}


\begin{table}[h]
	\caption{Categorization of related methods}
	\label{tab:categorization}
	\begin{center}
		\begin{tabular}{ c|c|c } 
			\hline
			Category & Top 1 & Top 3 \\\hline
			Complementary method &  20 (40\%) & 55 (37\%)\\\hline 
			Supplementary method &  18 (36\%) & 53 (35\%) \\ \hline
			Different implementation &  2 (4\%) & 4 (3\%)\\ \hline
			Not related & 10 (20\%) & 38 (25\%)\\\hline
			Total & 50 & 150 \\\hline
		\end{tabular}		
	\end{center}

\end{table}
		

\subsubsection{Complementary method} In this category, the query code can function alone, but the related method provides extra functionality to the query code and will further complete the user class. For the example shown as Listing \ref{lst:mot-query} and \ref{lst:mot-related} in Section~\ref{sec:intro}, the query snippet implements unzip a folder in Java.  We find {\ttt zip} function. These two methods can function independently, but often implemented together to get a stronger ability for file manipulation. 

Similarly, we find {\ttt decrypt} function for {\ttt encrypt} and {\ttt onPause} function for {\ttt onCreate} in Table \ref{tab:compl-examples}. The two methods in each pair do not have any direct function call association between them, but they complete each other with extra functionality and are often implemented together in real-life scenarios. 

For the 30 most common co-occurring methods, 40\% of them are complementary methods. 37\% of the sampled top 3 commonly co-occurring methods belong to this category.

\subsubsection{Supplementary method} The related code serves as a helper function to the query, or vice versa. One may make function call to the other. For example the {\ttt merge} function for {\ttt sort}. {\ttt sort} calls {\ttt merge} as a helper function and cannot achieve functionality without it. 

In our first example in Table \ref{tab:suppl-examples}, our related code {\ttt loadDrawable} calls {\ttt queueJob} inside its method body. 
There is another related method being recommended together with {\ttt loadDrawable}, which is shown below in Listings \ref{lst:part2}. {\ttt loadDrawable} also makes a function call to {\ttt getDrawableFromCache} inside its method body, The related methods give the user a broader picture of the whole class, point to a higher level of functionality the user may want to implement, and also direct the user to the most-frequently used higher level functionality and its auxiliaries.

More than one third of the sampled results are supplementary methods.

\begin{lstlisting}[caption={Recommended code \#2}, label={lst:part2}]
public static Drawable getDrawableFromCache(String url) {
	if (DrawableManager.cache.containsKey(url)) {
		return DrawableManager.cache.get(url);
	}
	
	return null;
}	
\end{lstlisting}

\subsubsection{Different implementation} This category represents those related methods that have similar functionality to the query code. The result provides an alternative, or a more detailed or extended implementation for the functionality. As shown in Table ~\ref{tab:diff-examples}, both of the methods implement sorting values in a {\ttt Map}, the query store the map entries in a {\ttt SortedSet}, while the related code uses {\ttt LinkedList}, and shows how to iterate a {\ttt Map}. For the {\ttt encrypt} function in Table ~\ref{tab:compl-examples}, the related code also provide an alternative implementation with {\ttt String} inputs, as shown in Listing ~\ref{lst:encryt}.

A small number of sampled methods provide different implementation to the query itself.

\begin{lstlisting}[caption={different implementation for \texttt{encrypt}}, label={lst:encryt}]
public static String encrypt(final String password, String message) throws GeneralSecurityException {
	try {
		final SecretKeySpec key = generateKey(password);
		log("message", message);
		byte[] cipherText = encrypt(key, ivBytes, message.getBytes(CHARSET));
		//NO_WRAP is important as was getting \n at the end
		String encoded = String.valueOf(
			Base64.encodeToString(cipherText, Base64.NO_PADDING ));
		log("Base64.NO_WRAP", encoded);
		return encoded;
	} catch (UnsupportedEncodingException e) {
		if (DEBUG_LOG_ENABLED)
			Log.e(TAG, "UnsupportedEncodingException ", e);
		throw new GeneralSecurityException(e);
	}
}
\end{lstlisting}

From the in-depth manual analysis from this section, we can see that there is a large amount of related code among the commonly co-occurring code, and they are worth to be considered for recommendation besides similar code to the query.


