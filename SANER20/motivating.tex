\begin{figure*}[!t]
\begin{minipage}[t]{0.5\linewidth}
\begin{lstlisting}[style=MyJavaSmallStyle, caption={Query code\todo{Simply listing 1 to a similar length as listing 2}}, label={lst:mot-query}]
public static boolean unpackZip(String path, String zipname, String targetDirectory) {
	InputStream is;
	ZipInputStream zis;
	try {
		String filename;
		is = new FileInputStream(path + zipname);
		zis = new ZipInputStream(new BufferedInputStream(is));
		ZipEntry ze;
		byte[] buffer = new byte[1024];
		int count;

		while ((ze = zis.getNextEntry()) != null) {
			filename = ze.getName();

			if (ze.isDirectory()) {
				File fmd = new File(targetDirectory + filename);
				fmd.mkdirs();
				continue;
			}

			FileOutputStream fout = new FileOutputStream(targetDirectory + filename);

			while ((count = zis.read(buffer)) != -1) {
				fout.write(buffer, 0, count);
			}

			fout.close();
			zis.closeEntry();
		}

		zis.close();
	} catch (IOException e) {
		e.printStackTrace();
		return false;
	}

	return true;
}
\end{lstlisting}
\end{minipage}
%
\begin{minipage}[t]{0.5\linewidth}
\begin{lstlisting}[style=MyJavaSmallStyle, caption={Recommended related code}, label={lst:mot-related}]
public static void zip(String baseFolder, List<File> files, String zipFile) {
	try  {
		BufferedInputStream origin = null;
		FileOutputStream dest = new FileOutputStream(zipFile);

		ZipOutputStream out = new ZipOutputStream(new BufferedOutputStream(dest));
		byte data[] = new byte[BUFFER];

		for (File file : files) {
			FileInputStream fi = new FileInputStream(file);
			origin = new BufferedInputStream(fi, BUFFER);
			String relativeFileName = file.getAbsolutePath().replace(baseFolder + File.separator , """");
			ZipEntry entry = new ZipEntry(relativeFileName);
			out.putNextEntry(entry);
			int count;
			while ((count = origin.read(data, 0, BUFFER)) != -1) {
				out.write(data, 0, count);
			}
			origin.close();
		}

		out.close();
	} catch(Exception e) {
		e.printStackTrace();
	}

}
\end{lstlisting}
\end{minipage}
\caption{An example of recommending relevant code that complements desired functionality}
\label{fig:example}
\end{figure*}