\subsection{Comparison with code search engines}
In this experiment we compare the recommendation results of \tool\ with those from code search engines. We focus on Google search and Stack Overflow search since these are popular destinations when people look for programming assistance. We also compare to \texttt{FaCoy}, a code-to-code-search engine which proved to have state-of-art precision. In the experiments above, we get 24 valid related code fragments as top 1 result recommended by \tool. We randomly select 6 snippets from them. For \texttt{FaCoy}, we use the snippets directly as search queries since it supports code-to-code search. For Google and SO search engine, we trace back to the SO post where the SO snippet originated, and summarize the context around the snippet as the search query. We show that whether the search engines can also retrieve top 1 related code fragments as \tool in their top 10 search results. 

Table \ref{tab:so-questions} lists the titles of SO posts used as search queries in our experiment.

\begin{table}
	\begin{center}
		\begin{tabular}{ c|c } 
			Query & Question title \\\hline 
			Q1 &  java unzip files from a specific folder\\\hline 
			Q2 &  2 \\ \hline
			Q3 &  3 \\ \hline
			Q4 &  4 \\ \hline
			Q5 &  5 \\ \hline
			Q6 &  6 \\ \hline
		\end{tabular}		
	\end{center}
	\caption{Queries for Google search and SO search}
	\label{tab:so-questions}
\end{table}

One example search query for \textit{FaCoy} is shown in List ~\ref{lst:facoy-query}.
\begin{lstlisting}[label={lst:facoy-query}]
	sample
\end{lstlisting}

\begin{table}
	\begin{center}
		\begin{tabular}{ c|c|c|c } 
			Query & Google search & SO search & FaCoy \\ 
			Q1 &  1  & &\\\hline 
			Q2 &  2 & &\\ \hline
			Q3 &  3 & &\\ \hline
			Q4 &  4 & &\\ \hline
			Q5 &  5 & & \\ \hline
			Q6 &  6 & &\\ \hline
		\end{tabular}		
	\end{center}
	\caption{Search results}
	\label{tab:so-questions}
\end{table}
