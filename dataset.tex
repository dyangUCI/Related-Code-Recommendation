\section{Data Collection}
\label{sec:dataset}
We apply our approach to Stack Overflow (SO) and GitHub. We use code snippets in SO as the pool of user-selected examples and use Java projects in GitHub as our code corpus to search from. We choose these two datasets not only because of their popularity within the programming community, but also because they are part of a larger system of software production. The same users that rely on the hosting and management characteristics of GitHub often have difficulties and need help on the implementation of their computer programs, seek support on SO for their specific problems, or hints of solutions from ones with a degree of similarity, and return to GitHub to apply the knowledge acquired. Previous work have shown that developers often copy and paste code snippets from Stack Overflow to their GitHub projects and make adaptations as needed~\cite{yang2017stack, an2017stack, wu2018developers, zhang2019analyzing}. Our approach will facilicate such opportunistic code reuse process when developers browse code snippets in Stack Overflow. The use scenario will be: when a user is interested in a code snippet in SO, CodeAid recommends related code fragments from GitHub, showing what other code she may also want to investigate and integrate into her own project. \todo{Di, if you have implemented the Chrome extension we discussed before, it would be great to add a screenshot of the Chrome extension here.} 


\subsection{GitHub}
We downloaded Java projects on GitHub by querying GHTorrent. GHTorrent is a scalable, offline mirror of data offered through the Github REST API, available to the research community as a service. It provides access to all the metadata of GitHub projects, e.g., the number of stars and commiters, main programming languages in a project, etc. Since GitHub has many toy projects that do not adequately reflect software engineering practices [19], we only consider GitHub projects that have at least five stars. To account for internal duplication in GitHub [20], we choose non-fork projects only and further remove duplicated GitHub files using the same file hashing method as in [20], since such file duplication may skew our analysis. As a result, we download 50,826 non-forked Java repositories with at least five stars from GitTorrent [21]. After deduplication, 5,825,727 distinct Java files remain.\todo{fix the citations.}


\subsection{Stack Overflow}
From the SO dump taken in October 2016, we extract 312,219 answer posts that have java or android tags and also contain code snippets in the {\ttt <code>} markdown. We consider code snippets in answer posts only, since snippets in question posts are rarely used as examples. Since SO snippets are often free-standing statements with low parsable rates, we used a customized pre-processor before tokenization. We add dummy class and method definitions, and semicolons after statements, as needed. For snippets contain multiple methods, we chunk them into individual ones. We keep only parsable SO snippets after pre-processing. Prior work finds that larger SO snippets have more meaningful clones in GitHub [26]. Hence, we choose to study SO examples with no less than 50 tokens after tokenization. We also remove duplicated examples within SO.\todo{fix the citations}

\subsection{Result for similar code detection}
We run SoucererCC to find all similar pairs between SO and GitHub. As a result, we get 21,207 distinct SO methods that have one or more similar code fragments in GitHub. \todo{talk more about the running enviroment and how long it takes to process the dataset, in order to show the construction effort.}

\subsection{Result for candidate related methods}
Within the 21,207 groups of SO snippet with GitHub files which contain similar methods to the SO snippet, we extract all co-occurred methods from these GitHub files and treat them as candidate related code fragments. Then for each candidate in each GitHub file, we retrieve its similar counterparts from other files. As a result, we get the co-occurred methods as our candidate related code fragments and for each candidate we also have the frequency of its similar counterparts. Not all groups have candidate methods and not all candidates have similar counterparts in other files, we have 11,110 SO snippets whose candidate related methods do have similar counterparts in other files, that can be taken as the candidate appears more than one files and we take this as a stronger signal for recommendation and only focus on these 11,110 groups from then on. Inside each group, we order the candidates by the number of similar counterparts and returned the ordered list as the final recommendation of related code fragments to the user. \todo{I suggest to add some statistics of these identified related methods. For example, how big are these methods in terms of LOC? For those SO snippets with related GitHub examples, how many related examples they have? Maybe we can also analyze the domains of SO snippets with SO, since I remember you mentioned that a lot of them are from Android.}

\todo{We also discussed that it is possible to recommend related code examples in another SO post if code snippets in two SO posts map to the same GitHub file. Would you still like to mention it here?}
